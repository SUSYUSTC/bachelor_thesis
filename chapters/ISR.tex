\section{Intermediate State Representation}

In Non-dyson ADC method, the eigenvalues of ADC secular matrix $\boldsymbol{K}+\boldsymbol{C}$ are exactly the ionization potentials, which behaves similar to Hamiltonian.
This feature suggests that $\boldsymbol{K}+\boldsymbol{C}$ is essentially a representation of Hamiltonian in some particular basis.
The surprising answer is that ADC secular equations can be deduced by a wavefunction approach, and the basis are refered to intermediate-state representation (ISR).
In addition, ISR approach allows us to analyze the origin of the compactness and size consistency of ADC.

We start from the so-called correlated excited (CE) states, which is defined by
\begin{equation}
	| \Psi_{J}^{0} \rangle=\hat{C}_{J} | \Psi_{0} \rangle
\end{equation}
where
\begin{equation}
	\left\{\hat{C}_{J}\right\}=\left\{c_{k} ; c_{a}^{\dagger} c_{k} c_{l}, k<l ; c_{a}^{\dagger} c_{b}^{\dagger} c_{j} c_{l} c_{l}, a<b, j<k<l ; \ldots\right\}
\end{equation}

Obviously, CE states are not orthogonal:
\begin{equation}
	S_{I J}=\langle\Psi_{I}^{0} | \Psi_{J}^{0}\rangle=\langle\Psi_{0}|\hat{C}_{I}^{\dagger} \hat{C}_{J}| \Psi_{0}\rangle \neq \delta_{I, J}
\end{equation}

Instead of the most obvious symmetric orthonormalization choice, which is 
\begin{equation}
	| \overline{\Psi}_{J} \rangle=\sum_{I} | \Psi_{I}^{0} \rangle\left(S^{-1 / 2}\right)_{I J}
\end{equation}
, we orthogonalize them by excitation class orthogonalization (ECO) procedure which states as follows:

\begin{itemize}
	\item First we construct intermediate states $\ket{\Psi^{\#}_K}$ by class. For states in class 1, we leave them unchanged, i.e.
		\begin{equation}
			\ket{\Psi^{\#}_i}=\ket{\Psi^0_i}
		\end{equation}
		Assuming intermediate states of classes 1, ..., $\nu-1$ have been constructed, then we orthogonalize $\ket{\tilde{\Psi}_J}$ in class $\nu$ by
		\begin{equation}
			| \Psi_{J}^{\#} \rangle=| \Psi_{J}^{0} \rangle-\sum_{[K]<\nu} | \Psi^{\#}_{K} \rangle\langle\Psi^{\#}_{K} | \Psi_{J}^{0}\rangle
		\end{equation}
	\item Orthogonalize states in each class by symmetric orthogonalization:
		\begin{equation}
			| \tilde{\Psi}_{J} \rangle=\sum_{[I]=\nu} | \Psi_{I}^{\#} \rangle\left(S_{\nu}^{-1 / 2}\right)_{I J}
		\end{equation}
\end{itemize}
where intermediate normalization is chosen.
\begin{equation}
	\langle\Psi_{0} | \Phi_{0}\rangle= 1
\end{equation}

Define ISR secular matrix $M$ by
\begin{equation}
	M_{I J}=\langle\tilde{\Psi}_{I}|\hat{H}-E_{0}| \tilde{\Psi}_{J}\rangle
\end{equation}
and 
\begin{equation}
	| \Psi_{n}^{N-1} \rangle=\sum_{I} X_{I n} | \tilde{\Psi}_{I} \rangle
\end{equation}

In matrix notation, the ISR secular equations take on the form
\begin{equation}
	\boldsymbol{M X}=\boldsymbol{X} \boldsymbol{\Omega}, \boldsymbol{X}^{\dagger} \boldsymbol{X}=\mathbf{1}
\end{equation}

Define ISR transition amplitudes as
\begin{equation}
	\tilde{f}_{I p}=\langle\tilde{\Psi}_{I}|c_{p}| \Psi_{0}\rangle
\end{equation}
with its relation to spectroscopic factor as
\begin{equation}
	x_{p}^{(n)}=\sum_{I} X_{I n}^{*} \tilde{f}_{I p}
\end{equation}

It can be proved that the ADC and ECO-ISR secular matrices are to be identified according to
\begin{equation}
	\begin{aligned} \boldsymbol{K}+\boldsymbol{C} & \equiv-\boldsymbol{M} \\ \boldsymbol{f} & \equiv \tilde{\boldsymbol{f}} \end{aligned}
\end{equation}

Although it is difficult to prove, one can check the second order of $M$ to verify it.
The verification is done in \cite{main}.

The equivalence of Non-dyson ADC and ISR approach results in two important features of ADC.
One is the compactness of matrix structure, which is the consequence of the so-called canonical perturbation theory order structure.
Compactness reduces the complexity of algorithm by reducing the size of total configuration space
The other is self consistency, which is guaranteed by the separability of ISR.
As is previously mentioned, self consistency ensures results of many important chemical processes calculation much more reliable.

Through out the calculation of $C$ and $f$ in each order, we find that they do not always begin from the zeroth order.
In fact, the leading order is dependent on the distance of the classes of two  indexes.
This rule states as follows:
\begin{equation}
	M_{I J} \sim O( | [I]-[J] | )
\end{equation}
which is also refered to as canonical order relations (COR).

For example, in CI matrix, the matrix element of class 1 and class 3 is in the first order, which breaks COR:
\begin{equation}
	H_{i, a b j k l}=\langle\Phi_{i}|\hat{H}_{I}| \Phi_{a b j k l}\rangle=-\delta_{i j} V_{k l[a b]}+\delta_{i k} V_{j I l[a b]}-\delta_{i l} V_{j k[a b]}
\end{equation}
while in the case of ISR,
\begin{equation}
	\begin{aligned} M_{i, a b j k l}^{(1)} &=\langle\tilde{\Psi}_{i}|\hat{H}-E_{0}| \tilde{\Psi}_{a b j k l}\rangle^{(1)} \\ &=H_{i, a b j k l}+\langle\tilde{\Psi}_{i}^{(1)}|\hat{H}_{0}-E_{0}^{(0)}| \Phi_{a b j k l}\rangle+\langle\Phi_{i}|\hat{H}_{0}-E_{0}^{(0)}| \tilde{\Psi}_{a b j k l}^{(1)}\rangle \end{aligned}
\end{equation}

According to ECO procedure,
\begin{equation}
	| \tilde{\Psi}_{i}^{(1)} \rangle=c_{i} | \Psi_{0}^{(1)} \rangle
\end{equation}
and
\begin{equation}
	\begin{aligned} | \tilde{\Psi}_{a b j k l}^{(1)} \rangle &=c_{a}^{\dagger} c_{b}^{\dagger} c_{j} c_{k} c_{l} | \Psi_{0}^{(1)} \rangle-| \Phi_{j} \rangle\langle\tilde{\Psi}_{j}^{(1)} | \Phi_{a b j k l}\rangle-| \Phi_{k} \rangle\langle\tilde{\Psi}_{k}^{(1)} | \Phi_{a b j k l}\rangle-| \Phi_{l} \rangle\langle\tilde{\Psi}_{l}^{(1)} | \Phi_{a b j k l}\rangle \\ &= c_{a}^{\dagger} c_{b}^{\dagger} c_{j} c_{k} c_{l} | \Psi_{0}^{(1)} \rangle-| \Phi_{j} \rangle v_{a b k l}^{*}+| \Phi_{k} \rangle_{a b j l}^{*}-| \Phi_{l} \rangle v_{a b j k}^{*} \end{aligned}
\end{equation}
where there are no first-order contributions from the orthogonalization with
respect to 2h-1p states.

Thus
\begin{equation}
	\langle\Phi_{i}|\hat{H}_{0}-E_{0}^{(0)}| \tilde{\Psi}_{a b j k l}^{(1)}\rangle=-\epsilon_{i}\langle\Phi_{i} | \tilde{\Psi}_{a b j k l}^{(1)}\rangle=\epsilon_{i}\left(\delta_{i j} v_{a b k l}^{*}-\delta_{i k} v_{a b j l}^{*}+\delta_{i l} v_{a b j k}^{*}\right)
\end{equation}
and
\begin{equation}
	\langle\tilde{\Psi}_{i}^{(1)}|\hat{H}_{0}-E_{0}^{(0)}| \Phi_{a b j k l}\rangle=\left(\delta_{i j} v_{a b k l}^{*}-\delta_{i k} v_{a b j l}^{*}+\delta_{i l} v_{a b j k}^{*}\right)\left(\epsilon_{a}+\epsilon_{b}-\epsilon_{j}-\epsilon_{k}-\epsilon_{l}\right)
\end{equation}

Adding these three up gives:
\begin{equation}
	M_{i, a b j k l}^{(1)}=0
\end{equation}

The general proof of COR for ADC is given in refernce \cite{main}.

Thus, if configuration space from class 1 to class $\mu$ is fully included, the truncation error order (TEO) of energies in 1h class is
\begin{equation}
	O_{T E}^{[n]}(\mu)=2\mu
\end{equation}
where TEO means the leading order of error.

A generalized form of this relation is TEO for in $n$th class is
\begin{equation} \label{generalTEO}
	O_{T E}^{[n]}(\mu)=2(\mu-n)
\end{equation}

While the CI case, 
\begin{equation}
H_{IJ}=\left\{ 
\begin{array}{ll}
	O(0), & \text{ } I=J
	\\
	O(1), & \text{ } |I-J|=1 \text{ or } 2
	\\
	0,    & \text{ } |I-J|>2
\end{array}
\right.
\end{equation}
which gives the TEO as
\begin{equation}
	O_{T E}(\mu)=\left\{\begin{array}{ll}{\mu,} & {\mu \text { even }} \\ {\mu+1,} & {\mu \text { odd }}\end{array}\right.
\end{equation}
which is a much larger error than ADC case.

Then we will discuss the size consistency of Non-dyson ADC.
Size consistency of Non-dyson ADC replies on two facts.

The first is factorization theorem, which states that intermediate state of system A+B is the direct product of that of A and B, i.e.
\begin{equation}
	| \tilde{\Psi}_{J_{A B}} \rangle=| \tilde{\Psi}_{J_{A}}^{A} \rangle | \tilde{\Psi}_{J_{B}}^{B} \rangle
\end{equation}
where $| \tilde{\Psi}_{J_{A B}} \rangle$ comes from
\begin{equation}
	| \Psi_{J_{A B}}^{0} \rangle=\hat{C}_{J_{A B}} | \Psi_{0} \rangle=\hat{C}_{J_{A}} | \Psi_{0}^{A} \rangle \hat{C}_{J_{B}} | \Psi_{0}^{B} \rangle
\end{equation}
and the same for $| \tilde{\Psi}_{J_{A}}^{0} \rangle$ and $| \tilde{\Psi}_{J_{B}}^{0} \rangle$.

The second is the separability of ADC secular matrix, which means excitation of only A, excitation of only B and excitation of both A and B are three independent diagonal blocks in ADC secular matrix, i.e. the following structure.
\begin{equation}
	\begin{array}{|c|c|c|}
		\hline
		M_{AA} & 0 & 0
		\\
		\hline
		0 & M_{BB} & 0
		\\
		\hline
		0 & 0 & M_{AB,AB}
		\\
		\hline
	\end{array}
\end{equation}

We will then prove each of them and also illustrate how do they relate to size consistency of ADC.

The full proof of factorization is in reference \cite{sizeproof}.
Then we take an example of 1h of A and 1h1p of B to see how does it work.
The corresponding CE state reads:
\begin{equation}
	| \Psi_{a^{\prime} j^{\prime} k}^{0} \rangle=c_{k} | \Psi_{0}^{A} \rangle c_{a^{\prime}}^{\dagger} c_{j^{\prime}} | \Psi_{0}^{B} \rangle
\end{equation}

The first step of ECO gives
\begin{equation}
	| \Psi_{a^{\prime} j^{\prime} k}^{\#} \rangle=| \Psi_{a^{\prime} j^{\prime} k}^{0} \rangle-\sum_{l} | \tilde{\Psi}_{l} \rangle\langle\tilde{\Psi}_{l} | \Psi_{a^{\prime} j^{\prime} k}^{0}\rangle
\end{equation}
where
\begin{equation}
	| \tilde{\Psi}_{l} \rangle=| \tilde{\Psi}_{l}^{A} \rangle | \Psi_{0}^{B} \rangle
\end{equation}

Thus
\begin{equation}
	\begin{aligned} 
		| \Psi_{a^{\prime} j^{\prime} k}^{\#} \rangle 
		&=c_{k} | \Psi_{0}^{A} \rangle c_{a^{\prime}}^{\dagger} c_{j^{\prime}} | \Psi_{0}^{B} \rangle-\sum_{l} | \tilde{\Psi}_{l}^{A} \rangle\langle\tilde{\Psi}_{l}^{A}|c_{k}| \Psi_{0}^{A}\rangle | \Psi_{0}^{B} \rangle\langle\Psi_{0}^{B}|c_{a^{\prime}}^{\dagger} c_{j^{\prime}}| \Psi_{0}^{B}\rangle 
		\\ 
		&= c_{k} | \Psi_{0}^{A} \rangle\left(c_{a^{\prime}}^{\dagger} c_{j^{\prime}} | \Psi_{0}^{B}\rangle-| \Psi_{0}^{B} \rangle\langle\Psi_{0}^{B}|c_{a^{\prime}}^{\dagger}| \Psi_{0}^{B}\rangle \right)
		\\
		&=| \Psi_{k}^{A \#} \rangle | \Psi_{a^{\prime} j^{\prime}}^{B \#} \rangle
	\end{aligned}
\end{equation}

Thus, after symmetric orthogonalization, the intermediate states of A+B will still be the direct product of that of A and B.
\begin{equation}
	| \tilde{\Psi}_{a^{\prime} j^{\prime} k} \rangle=| \tilde{\Psi}_{k}^{A} \rangle | \tilde{\Psi}_{a^{\prime} j^{\prime}}^{B} \rangle
\end{equation}

To prove the separability, we need to prove that $M_{I_A J_B}=0$, $M_{I_A, J_{AB}}=0$ and $M_{I_B J_{AB}}=0$.
Since A and B are treated on equal footing, we only need to prove the first two.
\begin{equation}
	M_{I_{A}, J_{B}}=\langle\tilde{\Psi}_{I_{A}}|\hat{H}| \tilde{\Psi}_{J_{B}}\rangle=\langle\tilde{\Psi}_{I_{A}}^{A}|\langle\Psi_{0}^{B}|\left(\hat{H}_{A}+\hat{H}_{B}\right)| \tilde{\Psi}_{J_{B}}^{B}\rangle| \Psi_{0}^{A}\rangle= 0
\end{equation}
\begin{equation}
	\begin{aligned}
		M_{I_{A}, J_{A B}} &=\langle\tilde{\Psi}_{I_{A}}|\hat{H}| \tilde{\Psi}_{J_{A B}}\rangle 
		\\
		&=\langle\tilde{\Psi}_{I_{A}}^{A}|\langle\Psi_{0}^{B}|\left(\hat{H}_{A}+\hat{H}_{B}\right)| \tilde{\Psi}_{J_{A}}^{A}\rangle| \tilde{\Psi}_{J_{B}}^{B}\rangle 
		\\
		&=\langle\tilde{\Psi}_{I_{S}}^{A} | \hat{H}_A | \tilde{\Psi}_{J_{A}}^{A}\rangle \langle\Psi_{0}^{B} | \tilde{\Psi}_{J_{B}}^{B}\rangle
		+\langle\tilde{\Psi}_{I_{A}}^{A} | \tilde{\Psi}_{J_{A}}^{A}\rangle \langle\Psi_{0}^{B}|\hat{H}_{B}| \tilde{\Psi}_{J_{B}}^{B}\rangle= 0
	\end{aligned}
\end{equation}

Having proved factorization theorem and separability of Non-dyson ADC, we will the illustrate its size consistency.
Size consistency requires that all eigenvalue of A should be reproduced by that of A+B, which requires Hamiltonian of A should not be coupled with B, and Hamiltonian of A in system of A should be equal to that of A+B.
Using factorization theorem and separability, the first requirement is already fulfilled.
The second requirement is equivalent to
%The size consistency requires the eigenvalues and eigenstates of A+B should be exactly the direct sum of those of A and B, which means
%\begin{equation}
%	\hat{H}_{A+B}=\hat{H}_A \otimes 1_B + 1_A \otimes \hat{H}_B
%\end{equation}
%
%By factorization theorem, this requirement is equivalent to
%\begin{equation}
%	\langle \tilde{\Psi}_{I_A} \tilde{\Psi}_{K_B} | \hat{H}_A + \hat{H}_B | \tilde{\Psi}_{J_A} \tilde{\Psi}_{L_B} \rangle
%	= \langle \tilde{\Psi}_{I_A} | \hat{H}_A | \tilde{\Psi}_{J_A} \rangle \langle \tilde{\Psi}_{K_B} | \tilde{\Psi}_{L_B} \rangle
%	+ \langle \tilde{\Psi}_{K_B} | \hat{H}_B | \tilde{\Psi}_{L_B} \rangle \langle \tilde{\Psi}_{I_A} | \tilde{\Psi}_{J_A} \rangle
%\end{equation}
\begin{equation}
	\langle \tilde{\Psi}_{I_A} \tilde{\Psi}_0^B | \hat{H}_{A+B}| \tilde{\Psi}_{J_A} \tilde{\Psi}_0^B \rangle
	= \langle \tilde{\Psi}_{I_A} | \hat{H}_A | \tilde{\Psi}_{J_A} \rangle
	+ E_0^B \delta_{IJ}
\end{equation}
which holds since $\ket{\tilde{\Psi}_{J_A} \tilde{\Psi}_0^B}$ is in the same excitation class with $\ket{\tilde{\Psi}_{J_A}}$. 

