% !TeX root = ../main.tex

\begin{abstract}
	在很长的一段时间里,电子亲和能一直是原子和分子的重要性质。
	尽管当前的实验方法可以以很高的精度测量电子亲和能,但对于例如化合物不稳定的组成部分的情况,实验方法并不总是适用,因而必须用理论方法去研究。
	但是,目前理论方法的发展严重滞后于实验手段,并且在很长的一段时间内没有明显的进展。
	事实上,目前的主要方法不是计算过于昂贵(例如耦合簇理论),就是不能得到令人满意的精度(例如密度泛函理论)。

	而除此之外,代数图构建则是另一个很有潜力的计算电子亲和能的量子化学方法。
	它在比耦合簇理论有着更低的计算成本的同时,有希望能够达到与其接近的计算精度。
	本质上,代数图构建是一个在多体场论框架下基于格林函数的方法。
	通过基于费曼图的微扰展开,代数图构建理论可以计算得到格林函数的各阶解析表达式,而这可以通过本征值问题与包括电子亲和能在内的很多重要的物理量联系在一起。
	受益于格林函数的微扰展开,代数图构建可以从戴森和非戴森两种方式得到。
	在戴森代数图构建理论中,通过利用自能的概念,对微扰项的求和可以部分地进行到无穷阶,但代价是在久期方程中阴离子和阳离子部分被耦合在了一起。
	而在非戴森代数图构建理论中,阴阳离子的耦合被解除,但求和到无穷阶的性质仍然被很好的保留。
	更进一步地,通过引入中间态表示,非戴森代数图构建理论可以从一种更简单的方式来理解,使得人们可以更好的分析其高效和具有尺寸一致性的原因。

	在我的本科的最后一个学期中,我在量子化学计算软件中有效的实现了二阶非戴森代数图构建的电子亲和能计算,并且做了基准化的计算。
	我将基准化计算的结果和单双激发态耦合簇理论进行了对比和全面的分析,并借此阐述了二阶代数图构建方法的优点和不足。
	

  \keywords{电子亲和能,量子化学,代数图构建,微扰论,格林函数,多体场论,中间态表示,尺寸一致性,程序实现,Q-Chem}
\end{abstract}

\begin{enabstract}
	Electron affinity (EA) has been an important property for atoms and molecules for a long time, and plays a very important role in many chemical processes.
	Although current experimental methods can measure EA of many molecules to very high accuracy, there are many cases like instable consituents that experiments do not apply and must be studied from a theoreical approach.
	However, current theoretical methods lag significantly behind experiments and have little advances to increase accuracy of EA calculation of large molecules for a long time.
	In fact, currently popular methods for EA either suffer from the problem of computationally expensive cost (coupled cluster), or cannot achieve satisfying accuracy (density functional theory).
	
	Another prominent quantum chemistry method for EA calculation is algebraic diagrammatic construction (ADC), which is not as expensive as coupled cluster (CC) while is potential to achieve accuracy of CC.
	Generally, ADC is based on a Green function formalism in the framework of many-body field theory.
	By perturbation expansion based on Feynman diagram scheme, it is possible to obtain analytical expressions for each order of Green function, which is related with many important physical quantities including EA by an eigenvalue problem.
	With the perturbation expansion of Green function, ADC can be constructed from a Dyson approach and a Non-dyson approach.
	In Dyson approach, the summation over perturbative terms can partially go to infinite orders by taking advantage of the concept self energy, with the cost to couple positive ion states and negative anion states in ADC secular equatioon.
	In Non-dyson approach, ion states and anion states are decoupled while the summation going to infinite orders is still well preserved.
	Furthermore, by introducing intermediate state representation (ISR), Non-dyson ADC can be understood from a simple way, which makes the analysis of its high performance and size consistency possible.

	In the last semester of undergraduate, I efficiently implemented second order of Non-dyson ADC of EA calculation in quantum chemistry program package Q-Chem and did a benchmarking calculation.
	The benchmarking calculation results are compared with CCSD and fully analyzed, and illustrated both advantages and disadvavntages of second order ADC method from the analysis.
  \enkeywords{electron affinity, quantum chemistry, algebraic diagrammatic contruction, perturbation theory, Green function, many-body field theory, intermediate state representation, size consistency, implementation, Q-Chem}
\end{enabstract}
