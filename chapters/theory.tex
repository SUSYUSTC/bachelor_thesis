\chapter{Theory of Algebraic Diagrammatic Construction}

Essentially, ADC is a kind of many-body perturbation theory, whose basic idea is to divide Hamiltonian to unperturbed part and perturbed part, and the sum over different orders of contribution.
If the summation is over all the contributions from infinite orders, then the exact energy will be obtained.
Obviously, numerically it is impossible to do so, thus a simple idea is to truncate the summation to some particular order.
However, when the order increases, on one hand, the expression for a direct perturbation become very complicated, on the other hand, the size inconsistency problem appears again.
Although it is shown that in the first few order all ill-behaved terms that break size consistency are canceled finally, a reason and proof is needed for higher orders.

On the other hand, instead of designed for solving ground state like what all the methods we discussed previously does, ADC is designed for ionization potential, electron affinity or excited state.
Interestingly, all these different purposes are based on a same theoretical framework, which is propagator, Green function and Feynman diagram.
There similarity is that all these processes include gain and loss of electrons, which is obvious in the ionization potential and electron affinity case.
In the excited state case, it can be viewed as gain of an electron with higher energy and loss with lower energy.
Thus, number of electrons is never conserved in ADC, which is hard to deal with in the formal quantum mechanics framework, which means a new tool is needed.

In order to finish these purposes, a many-body field theory approach is required, which originates from quantum field theory which is developed for a theoretical elementary particle physics.
After quantum field theory is constructed, the idea of field theory is quickly transferred to many-body physics, which becomes the basis of many-body perturbation theory.

In many-body field theory, a language of second quantization is used.
We have used a little second quantization when discussing the Post-Hartree Fock part for convenience.
However, we didn't give a formal definition for the notations we give.

Thus, in this chapter, we will firstly formally introduce second quantization, Green function and Feynman diagram.
Then we will discuss how these concepts are applied to ADC and to calculate the three kinds of energies mentioned above.
Finally, we will discuss the concept intermediate state and its relation with size consistency and compactness, and prove that ADC is both a canonical and size consistent method.


