% !TeX root = ../main.tex
\chapter{Development of Electronic Structure Theory}
Electronic structure usually refers to the quantum state of electrons in electromagnetic field produced by nuclei and the electrons themselves.\cite{elecstruc}
In principle, state of electrons and nuclei must be determined simultaneously by many-body Schrodinger equation.
However, the fact that nuclei are much heavier than electrons make the treatment of nuclei as stationary point charge a good approximation for ground state calculation of most quantum systems, which is the well-known Born-Oppenheimer approximation.\cite{sakurai}
The many-body Hamiltonian of electrons under Born-Oppenheimer approximation is
\begin{equation}
H=\sum_{i=1}^{N}\left(\frac{\boldsymbol{p}_{i}^{2}}{2 m}-\frac{Z e^{2}}{r_{i}}\right)+\sum_{i>j} \frac{e^{2}}{|\boldsymbol{x}_{i}-\boldsymbol{x}_{j}|}
\end{equation}
which contains a one-body part and a two-body part.
The two-body part couples all the electrons and make the exact solution of Schrodinger equation very difficult and almost impossible.
Thus, appropriate approximation are needed.


\section{Hartree-Fock Method}
The most basic and famous approximation among all of them is Hartree-Fock method.\cite{hartree}
Hartree-Fock method approximates the many-body wavefuntion as a single Slater determinant in which each row represents a molecular orbital. 
Usually the molecular orbitals are linear combinations of atomic orbitals. 
Then the coefficients are chosen to make the energy of the whole system as low as possible, which results in the following nonlinear Hartree-Fock Equation:
\begin{equation}
\mathbf{F} \mathbf{C}=\mathbf{S} \mathbf{C} \epsilon
\end{equation}
where
\begin{equation}
\hat{F}(i)=\hat{h}(i)+\sum_{j=1}^{n / 2}\left[2 \hat{J}_{j}(i)-\hat{K}_{j}(i)\right]
\end{equation}
is Fock matrix, 
\begin{equation}
\mathbf{S}_{j k}=\langle b_{j} | b_{k}\rangle=\int \Psi_{j}^{*} \Psi_{k} d \tau
\end{equation}
is Overlap matrix,
$\mathbf{C}$ is decomposition coefficients of molecular orbitals on atomic orbitals,
$\epsilon$ is energies of molecular orbitals.

The fact that $\mathbf{C}$ is implicitly contained in $\mathbf{F}$ make the Hartree-Fock Equation nonlinear and usually solved by Self-Consistent Field (SCF) method.

Essentially, Hartree-Fock is a mean-field theory which appropriate the electron-electron repulsion potential by averaging over the molecular orbitals.
Thus, instantaneous interactions between electrons are neglected, which make the decoupling between electrons possible.
All the effects arisen from deviations from the mean-field approximation are collectively used as a definition of electron correlation.

Under non-relativity and Born-Oppenheimer approximation, electron correlation is always the center topic of electronic structure.
Huge amount of efforts have been put on different methods of treatment of electron correlation,
including various Post-Hartree-Fock methods, Density Functional Theory and other less known methods like reduced density matrix \cite{reduceddm} and density matrix renormalization group. \cite{dmrg}

\section{Post-Hartree-Fock Methods}
Post-Hartree-Fock methods are the set of methods developed to improve on the Hartree-Fock method. They add at least part of electron correlation energy while Hartree-Fock neglects electron correlation completely.
Here electron correlation energy is defined as the energy difference between true ground state and Hartree-Fock ground state. Since Hartree-Fock energy is the upper bound to the exact energy, electron correlation energy is always negative.
Famous Post-Hartree-Fock methods includes Configuration Interaction (CI) \cite{mqc}, Coupled Cluster (CC) \cite{coupledcluster}, M{\o}ller–Plesset (MP) perturbation theory \cite{mp2}, and Algebraic Diagrammatic Construction (ADC).

\subsection{Configuration Interaction}
Among all the approaches to explore electron correlation energy, CI is conceptually the simplest, although not computationally the simplest.
The basic idea is to diagonalize the N-electron Hamiltonian in a basis N-electron function.
If all possible configuration in a particular basis set is used for linear combination, then CI reduces to full CI (FCI).
However, the computational cost of FCI is very large, since there will be $C_{2K}^{N}$ different Slater orbitals, which increases exponentially with the size of the system.
Thus, usually only part of the configurations are used in a CI calculation.

The CI wavefunction is usually written as:
\begin{equation}
\ket{\Psi_0}=t_0 \ket{\Phi_0} + \sum_{ia} t^{a}_{i}\ket{\Phi^{a}_{i}}+\sum_{ijab}t^{ab}_{ij}\ket{\Phi^{ab}_{ij}}+ \dots
\end{equation}
where $\ket{\Phi_0}$ is the Hartree-Fock ground state, and the following terms on right-hand side (RHS) of the equation are defined as single excitations, double excitations, etc.

In a symbolic form, the CI wavefunction can also be written as
\begin{equation} \label{CIexcite}
	\ket{\Psi_0}=t_0\ket{\Phi_0}+t_S\ket{S}+t_D\ket{D}+t_T\ket{T}+t_Q\ket{Q}+\dots
\end{equation}

The following is the matrix structure of CI Hamiltonian.
\begin{equation}
H=
\left[ 
\begin{array}{cccccc}
	\langle\Phi_{0}|H| \Phi_{0} \rangle & & & & & \dots
	\\ 
	0 & \langle S|H| S\rangle & & & & \dots
	\\
	\langle D|H| \Phi_{0} \rangle & \langle D|H| S\rangle & \langle D|H| D\rangle & & & \dots
	\\
	0 & \langle T|H| S\rangle & \langle T|H| D\rangle & \langle T|H| T\rangle & & \dots
	\\
	0 & 0 & \langle Q|H| D\rangle & \langle Q|H| T\rangle & \langle Q|H| Q\rangle & \dots
	\\
	{\vdots} & {\vdots} & {\vdots} & {\vdots} & {\vdots} & {\vdots}
\end{array}
\right]
\end{equation}

\subsection{Discussion of Size Consistency}
Size consistency is a concept relating to how the behaviour of quantum chemistry calculations changes with size. \cite{sizeconsistency}
Let A and B be two non-interacting systems.
If a given theory for the evaluation of the energy is size consistent, then the energy of the supersystem A+B, separated by a sufficiently large distance so there is essentially no shared electron density, is equal to the sum of the energy of A plus the energy of B taken by themselves ($E(A+B)=E(A)+E(B)$).
Size consistency is a vital behaviour for any method to calculate some properties like obtaining dissociation curves.

Unfortunately, standard CI is not size consistent, which is one of the main problems of CI.
To better understand the size inconsistency of CI, we can think about CI single (CIS) method, in which only single excitation is taken into account.
We further assume that both A and B are two-level systems.
When A and B are considered separately, excitation is allowed in both A and B.
However, when A and B is treated as a single system, excitation of A and B are not allowed to happen simultaneously, which will cause a difference in energy.

Another example is non-interacting $\text{H}_2$ molecules, which is usually used to test the size consistency of methods. \cite{h2problem}
CI gives the result that the limiting behaviour of total energy $E$ as N $\rightarrow \infty$ is $E \sim N^{-\frac{1}{2}}$, which is completely unreasonable.

In comparison, Coupled cluster and ADC are size consistent, thus are used much more than CI in calculations.

\subsection{Coupled Cluster}
The CI N-body wavefuntion \ref{CIexcite} can be rewritten as 
\begin{equation}
\ket{\Psi_0}=(t_0+T_1+T_2+T_3 \dots)\ket{\Phi_0}
\end{equation}
where the $T$ here are convenient to be expressed as second quantization form:
\begin{equation}
\begin{aligned}
	\hat{T}_{1} &=\sum_{i a} t_{i}^{a} \hat{c}_{a}^{\dagger} \hat{c}_{i}
	\\
	\hat{T}_{2} &=\frac{1}{4} \sum_{i j a b} t_{i j}^{a b} \hat{c}_{a}^{\dagger} \hat{c}_{b}^{\dagger} \hat{c}_{j} \hat{c}_{i}
	\\
	&\dots
	\\
	\hat{T}_{n}&=\left(\frac{1}{n !}\right)^{2} \sum_{i j \ldots a b \ldots}^{n} t_{i j \ldots}^{a b \ldots} \hat{c}_{a}^{\dagger} \hat{c}_{b}^{\dagger} \ldots \hat{c}_{j} \hat{c}_{i}
\end{aligned}
\end{equation}
with $\hat{c}^{\dagger}$ and $\hat{c}$ are creation operator and annihilation operation, respectively.

We have seen this formulation will cause size inconsistency.
Coupled cluster solves this problem by write N-body wavefuntion as
\begin{equation}
\ket{\Psi_0}=e^{t_0+T_1+T_2+T_3 \dots}\ket{\Phi_0}
\end{equation}
or usually
\begin{equation}
\ket{\Psi_0}=e^{T}\ket{\Phi_0}
\end{equation}
by defining
\begin{equation}
T=t_0+T_1+T_2+T_3 \dots
\end{equation}

Schrodinger equation gives
\begin{equation}
H | \Psi_{0} \rangle=H e^{T} | \Phi_{0} \rangle=E e^{T} | \Phi_{0} \rangle
\end{equation}
where the Hamiltonian can also be written as second quantization form
\begin{equation}
\hat{H}=\hat{K}+\hat{V}=\sum k_{p q} c_{p}^{\dagger} c_{q}+\frac{1}{2} \sum V_{p q r s} c_{p}^{\dagger} c_{q}^{\dagger} c_{s} c_{r}
\end{equation}
where K instead of T is used as kinetic part to avoid mixing with the coefficients t.

Thus
\begin{equation}
\begin{array}{l}
{\langle\Phi_{0}|e^{-T} H e^{T}| \Phi_{0}\rangle= E\langle\Phi_{0} | \Phi_{0}\rangle= E}
\\
{\langle\Phi^{*}|e^{-T} H e^{T}| \Phi_{0}\rangle= E\langle\Phi^{*} | \Phi_{0}\rangle= 0}
\end{array}
\end{equation}
where $\ket{\Phi^{*}}$ is any single, double, etc Hartree-Fock excited state, which is orthogonal to the Hartree-Fock ground state.

Then we truncate T to some order n, which is most usually chosen to 2, namely
\begin{equation}
T=t_0+T_1+T_2+ \dots +T_n
\end{equation}

Then we need to determine all the coefficients $t$ in the expressions of $T$ in eq.
Obviously, the required number of equations is equal to the number of variables $t$.
Since the number of degree of freedom of $T_i$ is equal to that of $i$-tuple excited state, we can just choose these equations as loop $\ket{\Phi^{*}}$ through single to n-tuple Hartree-Fock excited states.

Although the expansion of $e^T$ is infinite, only the first few terms have contribution to the final result of 
\begin{equation}
\langle\Phi^{*}|e^{-T} H e^{T}| \Phi_{0}\rangle
\end{equation}

This is because all orders of $T$ remove occupied electrons and add virtual electrons.
Thus, the summation of orders of all $T$ in expansion of $e^{-T} H e^{T}$ should be at most $i+2$ if the $\ket{\Phi^{*}}$ here is i-tuple excited state.
The contribution of an additional 2 orders is from $c_{p}^{\dagger} c_{q}^{\dagger} c_{s} c_{r}$ in Hamiltonian.

The resulting equations are a set of non-linear equations which are solved in an iterative manner. 
Standard quantum chemistry packages (GAMESS (US), NWChem, ACES II, etc.) solve the CC equations using the Jacobi method and direct inversion of the iterative subspace (DIIS) extrapolation of the t-amplitudes to accelerate convergence.

Unlike CI, CC is size consistent.
To better understand it, we still use the previous two-level systems A and B, CIS doesn't allow the simultaneous excitation of A and B, but in CCS, it is allowed by writing the final wavefuntion as 
\begin{equation}
\begin{aligned}
	\ket{\Psi_0}&=e^{T_A+T_B}\ket{\Phi_0}
	\\
	&=(1+T_A+T_B+\frac{1}{2}T_AT_B+\frac{1}{2}T_BT_A)\ket{\Phi_0}
	\\
	&=(1+T_A+T_B+T_AT_B)\ket{\Phi_0}
\end{aligned}
\end{equation}
where $T_A$ and $T_B$ commutes since there is no interaction between A and B.

\subsection{M{\o}ller–Plesset Perturbation Theory}
In quantum mechanics, a common way to describe complicated quantum system is perturbation theory approach.
The idea is to start with a simple unperturbed Hamiltonian for which a mathematical solution is known, and add an additional perturbed Hamiltonian representing a weak disturbance to the system.

The total Hamiltonian is
\begin{equation}
H=H_{0}+\lambda V
\end{equation}

Our goal is to express $E_n$ and $\ket{n}$  in terms of the energy levels and eigenstates of the old Hamiltonian:
\begin{equation}
\begin{aligned}
	E_{n}&=E_{n}^{(0)}+\lambda E_{n}^{(1)}+\lambda^{2} E_{n}^{(2)}+\cdots
	\\
	| n \rangle&=| n^{(0)} \rangle+\lambda | n^{(1)} \rangle+\lambda^{2} | n^{(2)} \rangle+\cdots
\end{aligned}
\end{equation}
where $H_0$ is the unperturbed Hamiltonian, while $V$ is the perturbed Hamiltonian.
$\lambda$ is a dimensionless parameter that can take on values ranging continuously from 0 (no perturbation) to 1 (the full perturbation)

In M{\o}ller–Plesset perturbation theory, the Hartree-Fock Hamiltonian is treated as unperturbed Hamiltonian while the rest part is treated as perturbed Hamiltonian:
\begin{equation}
v=\frac{1}{2} \sum V_{p q r s} c_{p}^{\dagger} c_{q}^{\dagger} c_{s} c_{r}-\sum_{k} V_{r k[s k]} n_{k} c_{r}^{\dagger} c_{s}
\end{equation}
where $n_k=1$ when the orbital k is occupied and $n_k=0$ when unoccupied, and $V_{ab[cd]}=V_{abcd}-V_{abdc}$

This kind of partition is called M{\o}ller–Plesset partition.

Up to second order perturbation theory, the expressions for the energy is
\begin{equation}
E_{n}(\lambda)=E_{n}^{(0)}+\lambda\langle n^{(0)}|v| n^{(0)}\rangle+\lambda^{2} \sum_{k \neq n} \frac{|\langle k^{(0)}|v| n^{(0)}\rangle|^{2}}{E_{n}^{(0)}-E_{k}^{(0)}}+O\left(\lambda^{3}\right)
\end{equation}
where the state $k$ and $n$ here are N-body state.

In fact, the first order contribution is zero:
\begin{equation}
\langle n^{(0)}|v| n^{(0)}\rangle=0
\end{equation}
which means Hartree-Fock energy is already accurate up to the first order, and MP2 is the leading term of the perturbation correction.

By replacing N-body state by Hartree-Fock single states and taking the anticommutation of wavefunction into account, we obtain the expression of MP2:
\begin{equation}
E_{\mathrm{MP} 2}=2 \sum_{i, j, a, b} \frac{\langle\varphi_{i} \varphi_{j}|\hat{v}| \varphi_{a} \varphi_{b}\rangle\langle\varphi_{a} \varphi_{b}|\hat{v}| \varphi_{i} \varphi_{j}\rangle}{\varepsilon_{i}+\varepsilon_{j}-\varepsilon_{a}-\varepsilon_{b}}-\sum_{i, j, a, b} \frac{\langle\varphi_{i} \varphi_{j}|\hat{v}| \varphi_{a} \varphi_{b}\rangle\langle\varphi_{a} \varphi_{b}|\hat{v}| \varphi_{j} \varphi_{i}\rangle}{\varepsilon_{i}+\varepsilon_{j}-\varepsilon_{a}-\varepsilon_{b}}
\end{equation}

Since both unperturbed and perturbed Hamiltonian have correct size scaling, MP2 is also a size consistent method.

\section{Density Functional Theory}

Unlike Post-Hartree-Fock methods, DFT does not start from Hartree-Fock result.
Instead, it uses a totally different approach, starting from the Hohenberg-Kohn theorem. \cite{hktheorem}
The theorem states that for a given ground state electron density distribution $\rho ({\vec{r}})$, the potential $V ({\vec{r})}$ is clearly defined and thus the ground state wave function $\Psi$.

Proof:
Assuming $\ket{\Psi_1}$ is the ground state of Hamiltonian $\hat{H}_1$ with external potential $V_1$, and $E_1$ is its energy.
\begin{equation}
E_{1}=\langle\Psi_{1}|\hat{H}_{1}| \Psi_{1}\rangle=\int V_{1}(\vec{r}) \rho(\vec{r})(\vec{r}) \mathrm{d}^{3} r+\langle\Psi_{1}|(\hat{T}+\hat{U})| \Psi_{1}\rangle
\end{equation}
where $\hat{T}$ is kinetic energy operator while $\hat{U}$ is coulomb operator.

If there is a different potential $V_2$ that leads to different ground state $\ket{\Psi_2}$:
\begin{equation}
E_{1}<\langle\Psi_{2}|\hat{H}_{1}| \Psi_{2}\rangle=\langle\Psi_{2}|\hat{H}_{2}| \Psi_{2}\rangle+\langle\Psi_{2}|\hat{H}_{1}-\hat{H}_{2}| \Psi_{2}\rangle= E_{2}+\int\left(V_{1}(\vec{r})-V_{2}(\vec{r})\right) \rho(\vec{r}) \mathrm{d}^{3} r
\end{equation}
Since potential $V_1$ and $V_2$ are treated on equal footing, the following inequality also satisfies:
\begin{equation}
E_{2}<\langle\Psi_{1}|\hat{H}_{2}| \Psi_{1}\rangle= E_{1}+\int\left(V_{2}(\vec{r})-V_{1}(\vec{r})\right) \rho(\vec{r}) \mathrm{d}^{3} r
\end{equation}
Adding these two inequalities gives:
\begin{equation}
E_{1}+E_{2}<E_{1}+E_{2}
\end{equation}
which means the second potential $V_2$ could not exist.

Thus
\begin{equation}
\Psi_{0}=\Psi[\rho_{0}]
\end{equation}
and
\begin{equation}
E_{0}=E[\rho_{0}]=\langle\Psi[\rho_{0}]|\hat{T}+\hat{V}+\hat{U}| \Psi[\rho_{0}]\rangle
\end{equation}
or
\begin{equation}
E_{s}[\rho]=\langle\Psi_{\mathrm{s}}[\rho]|\hat{T}+\hat{V}_{\mathrm{s}}| \Psi_{\mathrm{s}}[\rho]\rangle
\end{equation}

Thus the Schrodinger equation is 
\begin{equation}
\left[-\frac{\hbar^{2}}{2 m} \nabla^{2}+V_{\mathrm{s}}(\vec{r})\right] \varphi_{i}(\vec{r})=\varepsilon_{i} \varphi_{i}(\vec{r})
\end{equation}
where there exist a unique functional $V_s$
\begin{equation}
V_{\mathrm{s}}(\vec{r})=V(\vec{r})+\int \frac{e^{2} \rho_{\mathrm{s}}\left(\vec{r}^{\prime}\right)}{|\vec{r}-\vec{r}^{\prime}|} \mathrm{d}^{3} r^{\prime}+V_{\mathrm{XC}}\left[\rho_{\mathrm{s}}(\vec{r})\right]
\end{equation}
that can give correct ground state density and energy.
Here the $V_{\mathrm{XC}}$ is called the exchange-correlation potential, which includes all the many-body interactions.
Since the Hartree term and $V_{\mathrm{XC}}$ depend on $\rho{\vec{r}}$, the problem of solving the Kohn–Sham equation has to be done in a self-consistent iterative way.

However, for most of the systems except free electron gas \cite{homogeneous}, the exact expression for exchange-correlation functional is unknown.
Thus, approximations and fittings are needed to determine exchange-correlation functional.
Some famous exchange-correlation functional types includes local density approximation (LDA)
\begin{equation}
E_{\mathrm{XC}}^{\mathrm{LDA}}[n]=\int \varepsilon_{\mathrm{XC}}(n) n(\vec{r}) \mathrm{d}^{3} r
\end{equation}
local spin density approximation (LSDA) 
\begin{equation}
E_{\mathrm{XC}}^{\mathrm{LSDA}}\left[n_{\uparrow}, n_{\downarrow}\right]=\int \varepsilon_{\mathrm{XC}}\left(n_{\uparrow}, n_{\downarrow}\right) n(\vec{r}) \mathrm{d}^{3} r
\end{equation}
and generalized gradient approximation (GGA)
\begin{equation}
E_{\mathrm{XC}}^{\mathrm{GGA}}\left[n_{\uparrow}, n_{\downarrow}\right]=\int \varepsilon_{\mathrm{XC}}\left(n_{\uparrow}, n_{\downarrow}, \nabla n_{\uparrow}, \nabla n_{\downarrow}\right) n(\vec{r}) \mathrm{d}^{3} r
\end{equation}

Difficulties in expressing the exchange part of the energy can be relieved by including a component of the exact exchange energy calculated from Hartree–Fock theory.
Functionals of this type are known as hybrid functionals, including B3LYP \cite{b3lyp}, PBE0 \cite{pbe0} and HSE \cite{hse}.

\section{Relativistic and Non-adiabatic Quantum Chemistry}
In all above discussion of Hartree-Fock, Post-Hartree-Fock and density functional theory, non-relativity and Born-Oppenheimer assumptions are made.
However, there do exist cases that at least one of them break.

Relativistic effects are also neglected in Hartree-Fock, which is fine in systems that all atoms are light atoms, but can cause problems when heavy atoms are included. \cite{relativ}
Many methods have been suggested to partially or fully consider the relativistic effects, including relativistic density functional theory \cite{reladft}, Douglas-Kroll-Hess approximation \cite{dkh} and exact two component method \cite{x2c}.
Pseudopotential is also usually used by approximating the relativistic effect as an additional potential. \cite{relaqchem}

Although Born-Oppenheimer approximation behaves well in most of the cases, there are some cases that quantum nuclear effect plays a very important role, especially in molecules that hydrogen bond is important. \cite{tomqne}
One of the most effective theoretical method to study quantum nuclear effect is path integral molecular dynamics. \cite{pimd}

