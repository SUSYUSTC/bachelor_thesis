% !TeX encoding = UTF-8
% !TeX program = xelatex
% !TeX spellcheck = en_US

\documentclass[bachelor, english]{ustcthesis}
% doctor|master|bachelor [academic|professional] [chinese|english] [print|pdf]
% [super|numebers|authoryear]

\title{非戴森代数图构建理论与实现}
\author{孙嘉策}
\major{物理学}
\supervisor{Andrew Dreuw教授}
\entitle{Theory and Implementation of Nondyson Algebraic Diagrammatic Construction}
\enauthor{Jiace Sun}
\enmajor{Physics}
\ensupervisor{Prof. Andrew Dreuw}

\usepackage{graphicx}
\graphicspath{{figures/}}
\usepackage{booktabs}
\usepackage{longtable}
\usepackage[ruled,linesnumbered]{algorithm2e}
\usepackage{siunitx}
\usepackage{amsthm}
\usepackage{hyperref}
\usepackage{braket}
\usepackage{simpler-wick}
\usepackage{savesym}
\savesymbol{text}
\usepackage{feynman}
\restoresymbol{TXF}{text}




\DeclareRobustCommand\cs[1]{\texttt{\char`\\#1}}
\newcommand\pkg{\textsf}

\renewcommand\vec{\symbf}
\newcommand\mat{\symbf}
\newcommand\ts{\symbfsf}
\newcommand\real{\mathbf{R}}




\begin{document}

\maketitle
\makestatement

\frontmatter
% !TeX root = ../main.tex

\begin{abstract}
	在很长的一段时间里,电子亲和能一直是原子和分子的重要性质。
	尽管当前的实验方法可以以很高的精度测量电子亲和能,但对于例如化合物不稳定的组成部分的情况,实验方法并不总是适用,因而必须用理论方法去研究。
	但是,目前理论方法的发展严重滞后于实验手段,并且在很长的一段时间内没有明显的进展。
	事实上,目前的主要方法不是计算过于昂贵(例如耦合簇理论),就是不能得到令人满意的精度(例如密度泛函理论)。

	而除此之外,代数图构建则是另一个很有潜力的计算电子亲和能的量子化学方法。
	它在比耦合簇理论有着更低的计算成本的同时,有希望能够达到与其接近的计算精度。
	本质上,代数图构建是一个在多体场论框架下基于格林函数的方法。
	通过基于费曼图的微扰展开,代数图构建理论可以计算得到格林函数的各阶解析表达式,而这可以通过本征值问题与包括电子亲和能在内的很多重要的物理量联系在一起。
	受益于格林函数的微扰展开,代数图构建可以从戴森和非戴森两种方式得到。
	在戴森代数图构建理论中,通过利用自能的概念,对微扰项的求和可以部分地进行到无穷阶,但代价是在久期方程中阴离子和阳离子部分被耦合在了一起。
	而在非戴森代数图构建理论中,阴阳离子的耦合被解除,但求和到无穷阶的性质仍然被很好的保留。
	更进一步地,通过引入中间态表示,非戴森代数图构建理论可以从一种更简单的方式来理解,使得人们可以更好的分析其高效和具有尺寸一致性的原因。

	在我的本科的最后一个学期中,我在量子化学计算软件中有效的实现了二阶非戴森代数图构建的电子亲和能计算,并且做了基准化的计算。
	我将基准化计算的结果和单双激发态耦合簇理论进行了对比和全面的分析,并借此阐述了二阶代数图构建方法的优点和不足。
	

  \keywords{电子亲和能,量子化学,代数图构建,微扰论,格林函数,多体场论,中间态表示,尺寸一致性,程序实现,Q-Chem}
\end{abstract}

\begin{enabstract}
	Electron affinity (EA) has been an important property for atoms and molecules for a long time, and plays a very important role in many chemical processes.
	Although current experimental methods can measure EA of many molecules to very high accuracy, there are many cases like instable consituents that experiments do not apply and must be studied from a theoreical approach.
	However, current theoretical methods lag significantly behind experiments and have little advances to increase accuracy of EA calculation of large molecules for a long time.
	In fact, currently popular methods for EA either suffer from the problem of computationally expensive cost (coupled cluster), or cannot achieve satisfying accuracy (density functional theory).
	
	Another prominent quantum chemistry method for EA calculation is algebraic diagrammatic construction (ADC), which is not as expensive as coupled cluster (CC) while is potential to achieve accuracy of CC.
	Generally, ADC is based on a Green function formalism in the framework of many-body field theory.
	By perturbation expansion based on Feynman diagram scheme, it is possible to obtain analytical expressions for each order of Green function, which is related with many important physical quantities including EA by an eigenvalue problem.
	With the perturbation expansion of Green function, ADC can be constructed from a Dyson approach and a Non-dyson approach.
	In Dyson approach, the summation over perturbative terms can partially go to infinite orders by taking advantage of the concept self energy, with the cost to couple positive ion states and negative anion states in ADC secular equatioon.
	In Non-dyson approach, ion states and anion states are decoupled while the summation going to infinite orders is still well preserved.
	Furthermore, by introducing intermediate state representation (ISR), Non-dyson ADC can be understood from a simple way, which makes the analysis of its high performance and size consistency possible.

	In the last semester of undergraduate, I efficiently implemented second order of Non-dyson ADC of EA calculation in quantum chemistry program package Q-Chem and did a benchmarking calculation.
	The benchmarking calculation results are compared with CCSD and fully analyzed, and illustrated both advantages and disadvavntages of second order ADC method from the analysis.
  \enkeywords{electron affinity, quantum chemistry, algebraic diagrammatic contruction, perturbation theory, Green function, many-body field theory, intermediate state representation, size consistency, implementation, Q-Chem}
\end{enabstract}

\tableofcontents
% \listoffigures
% \listoftables
\input{chapters/notation}

\mainmatter
% !TeX root = ../main.tex
\chapter{Development of Electronic Structure Theory}
Electronic structure usually refers to the quantum state of electrons in electromagnetic field produced by nuclei and the electrons themselves.\cite{elecstruc}
In principle, state of electrons and nuclei must be determined simultaneously by many-body Schrodinger equation.
However, the fact that nuclei are much heavier than electrons make the treatment of nuclei as stationary point charge a good approximation for ground state calculation of most quantum systems, which is the well-known Born-Oppenheimer approximation.\cite{sakurai}
The many-body Hamiltonian of electrons under Born-Oppenheimer approximation is
\begin{equation}
H=\sum_{i=1}^{N}\left(\frac{\boldsymbol{p}_{i}^{2}}{2 m}-\frac{Z e^{2}}{r_{i}}\right)+\sum_{i>j} \frac{e^{2}}{|\boldsymbol{x}_{i}-\boldsymbol{x}_{j}|}
\end{equation}
which contains a one-body part and a two-body part.
The two-body part couples all the electrons and make the exact solution of Schrodinger equation very difficult and almost impossible.
Thus, appropriate approximation are needed.


\section{Hartree-Fock Method}
The most basic and famous approximation among all of them is Hartree-Fock method.\cite{hartree}
Hartree-Fock method approximates the many-body wavefuntion as a single Slater determinant in which each row represents a molecular orbital. 
Usually the molecular orbitals are linear combinations of atomic orbitals. 
Then the coefficients are chosen to make the energy of the whole system as low as possible, which results in the following nonlinear Hartree-Fock Equation:
\begin{equation}
\mathbf{F} \mathbf{C}=\mathbf{S} \mathbf{C} \epsilon
\end{equation}
where
\begin{equation}
\hat{F}(i)=\hat{h}(i)+\sum_{j=1}^{n / 2}\left[2 \hat{J}_{j}(i)-\hat{K}_{j}(i)\right]
\end{equation}
is Fock matrix, 
\begin{equation}
\mathbf{S}_{j k}=\langle b_{j} | b_{k}\rangle=\int \Psi_{j}^{*} \Psi_{k} d \tau
\end{equation}
is Overlap matrix,
$\mathbf{C}$ is decomposition coefficients of molecular orbitals on atomic orbitals,
$\epsilon$ is energies of molecular orbitals.

The fact that $\mathbf{C}$ is implicitly contained in $\mathbf{F}$ make the Hartree-Fock Equation nonlinear and usually solved by Self-Consistent Field (SCF) method.

Essentially, Hartree-Fock is a mean-field theory which appropriate the electron-electron repulsion potential by averaging over the molecular orbitals.
Thus, instantaneous interactions between electrons are neglected, which make the decoupling between electrons possible.
All the effects arisen from deviations from the mean-field approximation are collectively used as a definition of electron correlation.

Under non-relativity and Born-Oppenheimer approximation, electron correlation is always the center topic of electronic structure.
Huge amount of efforts have been put on different methods of treatment of electron correlation,
including various Post-Hartree-Fock methods, Density Functional Theory and other less known methods like reduced density matrix \cite{reduceddm} and density matrix renormalization group. \cite{dmrg}

\section{Post-Hartree-Fock Methods}
Post-Hartree-Fock methods are the set of methods developed to improve on the Hartree-Fock method. They add at least part of electron correlation energy while Hartree-Fock neglects electron correlation completely.
Here electron correlation energy is defined as the energy difference between true ground state and Hartree-Fock ground state. Since Hartree-Fock energy is the upper bound to the exact energy, electron correlation energy is always negative.
Famous Post-Hartree-Fock methods includes Configuration Interaction (CI) \cite{mqc}, Coupled Cluster (CC) \cite{coupledcluster}, M{\o}ller–Plesset (MP) perturbation theory \cite{mp2}, and Algebraic Diagrammatic Construction (ADC).

\subsection{Configuration Interaction}
Among all the approaches to explore electron correlation energy, CI is conceptually the simplest, although not computationally the simplest.
The basic idea is to diagonalize the N-electron Hamiltonian in a basis N-electron function.
If all possible configuration in a particular basis set is used for linear combination, then CI reduces to full CI (FCI).
However, the computational cost of FCI is very large, since there will be $C_{2K}^{N}$ different Slater orbitals, which increases exponentially with the size of the system.
Thus, usually only part of the configurations are used in a CI calculation.

The CI wavefunction is usually written as:
\begin{equation}
\ket{\Psi_0}=t_0 \ket{\Phi_0} + \sum_{ia} t^{a}_{i}\ket{\Phi^{a}_{i}}+\sum_{ijab}t^{ab}_{ij}\ket{\Phi^{ab}_{ij}}+ \dots
\end{equation}
where $\ket{\Phi_0}$ is the Hartree-Fock ground state, and the following terms on right-hand side (RHS) of the equation are defined as single excitations, double excitations, etc.

In a symbolic form, the CI wavefunction can also be written as
\begin{equation} \label{CIexcite}
	\ket{\Psi_0}=t_0\ket{\Phi_0}+t_S\ket{S}+t_D\ket{D}+t_T\ket{T}+t_Q\ket{Q}+\dots
\end{equation}

The following is the matrix structure of CI Hamiltonian.
\begin{equation}
H=
\left[ 
\begin{array}{cccccc}
	\langle\Phi_{0}|H| \Phi_{0} \rangle & & & & & \dots
	\\ 
	0 & \langle S|H| S\rangle & & & & \dots
	\\
	\langle D|H| \Phi_{0} \rangle & \langle D|H| S\rangle & \langle D|H| D\rangle & & & \dots
	\\
	0 & \langle T|H| S\rangle & \langle T|H| D\rangle & \langle T|H| T\rangle & & \dots
	\\
	0 & 0 & \langle Q|H| D\rangle & \langle Q|H| T\rangle & \langle Q|H| Q\rangle & \dots
	\\
	{\vdots} & {\vdots} & {\vdots} & {\vdots} & {\vdots} & {\vdots}
\end{array}
\right]
\end{equation}

\subsection{Discussion of Size Consistency}
Size consistency is a concept relating to how the behaviour of quantum chemistry calculations changes with size. \cite{sizeconsistency}
Let A and B be two non-interacting systems.
If a given theory for the evaluation of the energy is size consistent, then the energy of the supersystem A+B, separated by a sufficiently large distance so there is essentially no shared electron density, is equal to the sum of the energy of A plus the energy of B taken by themselves ($E(A+B)=E(A)+E(B)$).
Size consistency is a vital behaviour for any method to calculate some properties like obtaining dissociation curves.

Unfortunately, standard CI is not size consistent, which is one of the main problems of CI.
To better understand the size inconsistency of CI, we can think about CI single (CIS) method, in which only single excitation is taken into account.
We further assume that both A and B are two-level systems.
When A and B are considered separately, excitation is allowed in both A and B.
However, when A and B is treated as a single system, excitation of A and B are not allowed to happen simultaneously, which will cause a difference in energy.

Another example is non-interacting $\text{H}_2$ molecules, which is usually used to test the size consistency of methods. \cite{h2problem}
CI gives the result that the limiting behaviour of total energy $E$ as N $\rightarrow \infty$ is $E \sim N^{-\frac{1}{2}}$, which is completely unreasonable.

In comparison, Coupled cluster and ADC are size consistent, thus are used much more than CI in calculations.

\subsection{Coupled Cluster}
The CI N-body wavefuntion \ref{CIexcite} can be rewritten as 
\begin{equation}
\ket{\Psi_0}=(t_0+T_1+T_2+T_3 \dots)\ket{\Phi_0}
\end{equation}
where the $T$ here are convenient to be expressed as second quantization form:
\begin{equation}
\begin{aligned}
	\hat{T}_{1} &=\sum_{i a} t_{i}^{a} \hat{c}_{a}^{\dagger} \hat{c}_{i}
	\\
	\hat{T}_{2} &=\frac{1}{4} \sum_{i j a b} t_{i j}^{a b} \hat{c}_{a}^{\dagger} \hat{c}_{b}^{\dagger} \hat{c}_{j} \hat{c}_{i}
	\\
	&\dots
	\\
	\hat{T}_{n}&=\left(\frac{1}{n !}\right)^{2} \sum_{i j \ldots a b \ldots}^{n} t_{i j \ldots}^{a b \ldots} \hat{c}_{a}^{\dagger} \hat{c}_{b}^{\dagger} \ldots \hat{c}_{j} \hat{c}_{i}
\end{aligned}
\end{equation}
with $\hat{c}^{\dagger}$ and $\hat{c}$ are creation operator and annihilation operation, respectively.

We have seen this formulation will cause size inconsistency.
Coupled cluster solves this problem by write N-body wavefuntion as
\begin{equation}
\ket{\Psi_0}=e^{t_0+T_1+T_2+T_3 \dots}\ket{\Phi_0}
\end{equation}
or usually
\begin{equation}
\ket{\Psi_0}=e^{T}\ket{\Phi_0}
\end{equation}
by defining
\begin{equation}
T=t_0+T_1+T_2+T_3 \dots
\end{equation}

Schrodinger equation gives
\begin{equation}
H | \Psi_{0} \rangle=H e^{T} | \Phi_{0} \rangle=E e^{T} | \Phi_{0} \rangle
\end{equation}
where the Hamiltonian can also be written as second quantization form
\begin{equation}
\hat{H}=\hat{K}+\hat{V}=\sum k_{p q} c_{p}^{\dagger} c_{q}+\frac{1}{2} \sum V_{p q r s} c_{p}^{\dagger} c_{q}^{\dagger} c_{s} c_{r}
\end{equation}
where K instead of T is used as kinetic part to avoid mixing with the coefficients t.

Thus
\begin{equation}
\begin{array}{l}
{\langle\Phi_{0}|e^{-T} H e^{T}| \Phi_{0}\rangle= E\langle\Phi_{0} | \Phi_{0}\rangle= E}
\\
{\langle\Phi^{*}|e^{-T} H e^{T}| \Phi_{0}\rangle= E\langle\Phi^{*} | \Phi_{0}\rangle= 0}
\end{array}
\end{equation}
where $\ket{\Phi^{*}}$ is any single, double, etc Hartree-Fock excited state, which is orthogonal to the Hartree-Fock ground state.

Then we truncate T to some order n, which is most usually chosen to 2, namely
\begin{equation}
T=t_0+T_1+T_2+ \dots +T_n
\end{equation}

Then we need to determine all the coefficients $t$ in the expressions of $T$ in eq.
Obviously, the required number of equations is equal to the number of variables $t$.
Since the number of degree of freedom of $T_i$ is equal to that of $i$-tuple excited state, we can just choose these equations as loop $\ket{\Phi^{*}}$ through single to n-tuple Hartree-Fock excited states.

Although the expansion of $e^T$ is infinite, only the first few terms have contribution to the final result of 
\begin{equation}
\langle\Phi^{*}|e^{-T} H e^{T}| \Phi_{0}\rangle
\end{equation}

This is because all orders of $T$ remove occupied electrons and add virtual electrons.
Thus, the summation of orders of all $T$ in expansion of $e^{-T} H e^{T}$ should be at most $i+2$ if the $\ket{\Phi^{*}}$ here is i-tuple excited state.
The contribution of an additional 2 orders is from $c_{p}^{\dagger} c_{q}^{\dagger} c_{s} c_{r}$ in Hamiltonian.

The resulting equations are a set of non-linear equations which are solved in an iterative manner. 
Standard quantum chemistry packages (GAMESS (US), NWChem, ACES II, etc.) solve the CC equations using the Jacobi method and direct inversion of the iterative subspace (DIIS) extrapolation of the t-amplitudes to accelerate convergence.

Unlike CI, CC is size consistent.
To better understand it, we still use the previous two-level systems A and B, CIS doesn't allow the simultaneous excitation of A and B, but in CCS, it is allowed by writing the final wavefuntion as 
\begin{equation}
\begin{aligned}
	\ket{\Psi_0}&=e^{T_A+T_B}\ket{\Phi_0}
	\\
	&=(1+T_A+T_B+\frac{1}{2}T_AT_B+\frac{1}{2}T_BT_A)\ket{\Phi_0}
	\\
	&=(1+T_A+T_B+T_AT_B)\ket{\Phi_0}
\end{aligned}
\end{equation}
where $T_A$ and $T_B$ commutes since there is no interaction between A and B.

\subsection{M{\o}ller–Plesset Perturbation Theory}
In quantum mechanics, a common way to describe complicated quantum system is perturbation theory approach.
The idea is to start with a simple unperturbed Hamiltonian for which a mathematical solution is known, and add an additional perturbed Hamiltonian representing a weak disturbance to the system.

The total Hamiltonian is
\begin{equation}
H=H_{0}+\lambda V
\end{equation}

Our goal is to express $E_n$ and $\ket{n}$  in terms of the energy levels and eigenstates of the old Hamiltonian:
\begin{equation}
\begin{aligned}
	E_{n}&=E_{n}^{(0)}+\lambda E_{n}^{(1)}+\lambda^{2} E_{n}^{(2)}+\cdots
	\\
	| n \rangle&=| n^{(0)} \rangle+\lambda | n^{(1)} \rangle+\lambda^{2} | n^{(2)} \rangle+\cdots
\end{aligned}
\end{equation}
where $H_0$ is the unperturbed Hamiltonian, while $V$ is the perturbed Hamiltonian.
$\lambda$ is a dimensionless parameter that can take on values ranging continuously from 0 (no perturbation) to 1 (the full perturbation)

In M{\o}ller–Plesset perturbation theory, the Hartree-Fock Hamiltonian is treated as unperturbed Hamiltonian while the rest part is treated as perturbed Hamiltonian:
\begin{equation}
v=\frac{1}{2} \sum V_{p q r s} c_{p}^{\dagger} c_{q}^{\dagger} c_{s} c_{r}-\sum_{k} V_{r k[s k]} n_{k} c_{r}^{\dagger} c_{s}
\end{equation}
where $n_k=1$ when the orbital k is occupied and $n_k=0$ when unoccupied, and $V_{ab[cd]}=V_{abcd}-V_{abdc}$

This kind of partition is called M{\o}ller–Plesset partition.

Up to second order perturbation theory, the expressions for the energy is
\begin{equation}
E_{n}(\lambda)=E_{n}^{(0)}+\lambda\langle n^{(0)}|v| n^{(0)}\rangle+\lambda^{2} \sum_{k \neq n} \frac{|\langle k^{(0)}|v| n^{(0)}\rangle|^{2}}{E_{n}^{(0)}-E_{k}^{(0)}}+O\left(\lambda^{3}\right)
\end{equation}
where the state $k$ and $n$ here are N-body state.

In fact, the first order contribution is zero:
\begin{equation}
\langle n^{(0)}|v| n^{(0)}\rangle=0
\end{equation}
which means Hartree-Fock energy is already accurate up to the first order, and MP2 is the leading term of the perturbation correction.

By replacing N-body state by Hartree-Fock single states and taking the anticommutation of wavefunction into account, we obtain the expression of MP2:
\begin{equation}
E_{\mathrm{MP} 2}=2 \sum_{i, j, a, b} \frac{\langle\varphi_{i} \varphi_{j}|\hat{v}| \varphi_{a} \varphi_{b}\rangle\langle\varphi_{a} \varphi_{b}|\hat{v}| \varphi_{i} \varphi_{j}\rangle}{\varepsilon_{i}+\varepsilon_{j}-\varepsilon_{a}-\varepsilon_{b}}-\sum_{i, j, a, b} \frac{\langle\varphi_{i} \varphi_{j}|\hat{v}| \varphi_{a} \varphi_{b}\rangle\langle\varphi_{a} \varphi_{b}|\hat{v}| \varphi_{j} \varphi_{i}\rangle}{\varepsilon_{i}+\varepsilon_{j}-\varepsilon_{a}-\varepsilon_{b}}
\end{equation}

Since both unperturbed and perturbed Hamiltonian have correct size scaling, MP2 is also a size consistent method.

\section{Density Functional Theory}

Unlike Post-Hartree-Fock methods, DFT does not start from Hartree-Fock result.
Instead, it uses a totally different approach, starting from the Hohenberg-Kohn theorem. \cite{hktheorem}
The theorem states that for a given ground state electron density distribution $\rho ({\vec{r}})$, the potential $V ({\vec{r})}$ is clearly defined and thus the ground state wave function $\Psi$.

Proof:
Assuming $\ket{\Psi_1}$ is the ground state of Hamiltonian $\hat{H}_1$ with external potential $V_1$, and $E_1$ is its energy.
\begin{equation}
E_{1}=\langle\Psi_{1}|\hat{H}_{1}| \Psi_{1}\rangle=\int V_{1}(\vec{r}) \rho(\vec{r})(\vec{r}) \mathrm{d}^{3} r+\langle\Psi_{1}|(\hat{T}+\hat{U})| \Psi_{1}\rangle
\end{equation}
where $\hat{T}$ is kinetic energy operator while $\hat{U}$ is coulomb operator.

If there is a different potential $V_2$ that leads to different ground state $\ket{\Psi_2}$:
\begin{equation}
E_{1}<\langle\Psi_{2}|\hat{H}_{1}| \Psi_{2}\rangle=\langle\Psi_{2}|\hat{H}_{2}| \Psi_{2}\rangle+\langle\Psi_{2}|\hat{H}_{1}-\hat{H}_{2}| \Psi_{2}\rangle= E_{2}+\int\left(V_{1}(\vec{r})-V_{2}(\vec{r})\right) \rho(\vec{r}) \mathrm{d}^{3} r
\end{equation}
Since potential $V_1$ and $V_2$ are treated on equal footing, the following inequality also satisfies:
\begin{equation}
E_{2}<\langle\Psi_{1}|\hat{H}_{2}| \Psi_{1}\rangle= E_{1}+\int\left(V_{2}(\vec{r})-V_{1}(\vec{r})\right) \rho(\vec{r}) \mathrm{d}^{3} r
\end{equation}
Adding these two inequalities gives:
\begin{equation}
E_{1}+E_{2}<E_{1}+E_{2}
\end{equation}
which means the second potential $V_2$ could not exist.

Thus
\begin{equation}
\Psi_{0}=\Psi[\rho_{0}]
\end{equation}
and
\begin{equation}
E_{0}=E[\rho_{0}]=\langle\Psi[\rho_{0}]|\hat{T}+\hat{V}+\hat{U}| \Psi[\rho_{0}]\rangle
\end{equation}
or
\begin{equation}
E_{s}[\rho]=\langle\Psi_{\mathrm{s}}[\rho]|\hat{T}+\hat{V}_{\mathrm{s}}| \Psi_{\mathrm{s}}[\rho]\rangle
\end{equation}

Thus the Schrodinger equation is 
\begin{equation}
\left[-\frac{\hbar^{2}}{2 m} \nabla^{2}+V_{\mathrm{s}}(\vec{r})\right] \varphi_{i}(\vec{r})=\varepsilon_{i} \varphi_{i}(\vec{r})
\end{equation}
where there exist a unique functional $V_s$
\begin{equation}
V_{\mathrm{s}}(\vec{r})=V(\vec{r})+\int \frac{e^{2} \rho_{\mathrm{s}}\left(\vec{r}^{\prime}\right)}{|\vec{r}-\vec{r}^{\prime}|} \mathrm{d}^{3} r^{\prime}+V_{\mathrm{XC}}\left[\rho_{\mathrm{s}}(\vec{r})\right]
\end{equation}
that can give correct ground state density and energy.
Here the $V_{\mathrm{XC}}$ is called the exchange-correlation potential, which includes all the many-body interactions.
Since the Hartree term and $V_{\mathrm{XC}}$ depend on $\rho{\vec{r}}$, the problem of solving the Kohn–Sham equation has to be done in a self-consistent iterative way.

However, for most of the systems except free electron gas \cite{homogeneous}, the exact expression for exchange-correlation functional is unknown.
Thus, approximations and fittings are needed to determine exchange-correlation functional.
Some famous exchange-correlation functional types includes local density approximation (LDA)
\begin{equation}
E_{\mathrm{XC}}^{\mathrm{LDA}}[n]=\int \varepsilon_{\mathrm{XC}}(n) n(\vec{r}) \mathrm{d}^{3} r
\end{equation}
local spin density approximation (LSDA) 
\begin{equation}
E_{\mathrm{XC}}^{\mathrm{LSDA}}\left[n_{\uparrow}, n_{\downarrow}\right]=\int \varepsilon_{\mathrm{XC}}\left(n_{\uparrow}, n_{\downarrow}\right) n(\vec{r}) \mathrm{d}^{3} r
\end{equation}
and generalized gradient approximation (GGA)
\begin{equation}
E_{\mathrm{XC}}^{\mathrm{GGA}}\left[n_{\uparrow}, n_{\downarrow}\right]=\int \varepsilon_{\mathrm{XC}}\left(n_{\uparrow}, n_{\downarrow}, \nabla n_{\uparrow}, \nabla n_{\downarrow}\right) n(\vec{r}) \mathrm{d}^{3} r
\end{equation}

Difficulties in expressing the exchange part of the energy can be relieved by including a component of the exact exchange energy calculated from Hartree–Fock theory.
Functionals of this type are known as hybrid functionals, including B3LYP \cite{b3lyp}, PBE0 \cite{pbe0} and HSE \cite{hse}.

\section{Relativistic and Non-adiabatic Quantum Chemistry}
In all above discussion of Hartree-Fock, Post-Hartree-Fock and density functional theory, non-relativity and Born-Oppenheimer assumptions are made.
However, there do exist cases that at least one of them break.

Relativistic effects are also neglected in Hartree-Fock, which is fine in systems that all atoms are light atoms, but can cause problems when heavy atoms are included. \cite{relativ}
Many methods have been suggested to partially or fully consider the relativistic effects, including relativistic density functional theory \cite{reladft}, Douglas-Kroll-Hess approximation \cite{dkh} and exact two component method \cite{x2c}.
Pseudopotential is also usually used by approximating the relativistic effect as an additional potential. \cite{relaqchem}

Although Born-Oppenheimer approximation behaves well in most of the cases, there are some cases that quantum nuclear effect plays a very important role, especially in molecules that hydrogen bond is important. \cite{tomqne}
One of the most effective theoretical method to study quantum nuclear effect is path integral molecular dynamics. \cite{pimd}


\chapter{Theory of Algebraic Diagrammatic Construction}

Essentially, ADC is a kind of many-body perturbation theory, whose basic idea is to divide Hamiltonian to unperturbed part and perturbed part, and the sum over different orders of contribution.
If the summation is over all the contributions from infinite orders, then the exact energy will be obtained.
Obviously, numerically it is impossible to do so, thus a simple idea is to truncate the summation to some particular order.
However, when the order increases, on one hand, the expression for a direct perturbation become very complicated, on the other hand, the size inconsistency problem appears again.
Although it is shown that in the first few order all ill-behaved terms that break size consistency are canceled finally, a reason and proof is needed for higher orders.

On the other hand, instead of designed for solving ground state like what all the methods we discussed previously does, ADC is designed for ionization potential, electron affinity or excited state.
Interestingly, all these differnt purposes are based on a same theoretical framework, which is propagator, Green function and Feynman diagram.
There similarity is that all these processes include gain and loss of electrons, which is obvious in the ionization potential and electron affinity case.
In the excited state case, it can be viewed as gain of an electron with higher energy and loss with lower energy.
Thus, number of electrons is never conserved in ADC, which is hard to deal with in the formal quantum mechanics framework, which means a new tool is needed.

In order to finish these purposes, a many-body field theory approach is required, which originates from quantum field theory which is developed for a theoretical elementary particle physics.
After quantum field theory is constructed, the idea of field theory is quickly transfered to many-body physics, which becomes the basis of many-body perturbation theory.

In many-body field theory, a language of second quantization is used.
We have used a little second quantization when discussing the Post-Hartree Fock part for convenience.
However, we didn't give a formal definition for the notations we give.

Thus, in this chapter, we will firstly formally introduce second quantization, Green function and Feynman diagram.
Then we will discuss how these concepts are applied to ADC and to calculate the three kinds of energies mentioned above.
Finally, we will discuss the concept intermediate state and its relation with size consistency, and prove that ADC is both a canonical and size consistent method.

% !TeX root = ../main.tex
\section{Theoretical backgroud of ADC}

\subsection{Identity particle and second quantization}
Let $\ket{q}$ be a set of orthonomal one particle basis
\begin{equation}
\begin{aligned}
\left\langle q | q^{\prime}\right\rangle &=\delta_{q q^{\prime}} \\ \sum_{q} | q \rangle\langle q |&=1
\end{aligned}
\end{equation}

Define $| q_{1} \ldots q_{N} \rangle$ as the state of $| q_{1} \rangle \dots | q_{N} \rangle$ after antisymmetrization and normalization:
\begin{equation}
| q_{1} \ldots q_{N} \rangle =(N !)^{-\frac{1}{2}} \sum_{P}(-1)^{P} | q_{P(1)} \rangle \ldots | q_{P(N)} \rangle
\end{equation}
where $P$ is permutation of indexes.

In field theory framework, the particle number is subjected to change.
Thus we should consider Fock space rather than Hilbert space.
Fock space is defined by
\begin{equation}
\mathcal{F}=\mathcal{H}_{0} \oplus \mathcal{H}_{1} \oplus \mathcal{H}_{2}^{A} \oplus \cdots
\end{equation}
where only antisymmetric states is included when $N>2$.

Then we can define creation operators by 
\begin{equation}
c_{q}^{\dagger} | q_{1} \ldots q_{N} \rangle=| q_{1} \ldots q_{N} q \rangle
\end{equation}
and annihilation operator by its hermitian adjoint.

Thus it's easy to calculate the annihilation operator as
\begin{equation}
\begin{aligned}
	c_{q} | q_{1} \ldots q_{N} \rangle&=\sum_{N^{\prime}} \sum_{q_{1}^{\prime}<q_{2}^{\prime} \cdots<q_{N^{\prime}}^{\prime}} | q_{1}^{\prime} \ldots q_{N^{\prime}}^{\prime} \rangle\left\langle q_{1}^{\prime} \ldots q_{N^{\prime}}^{\prime}\left|c_{q}\right| q_{1} \ldots q_{N}\right\rangle
	\\
	&=\sum_{N^{\prime}} \sum_{q_{1}^{\prime}<q_{2}^{\prime} \cdots<q_{N^{\prime}}^{\prime}} | q_{1}^{\prime} \ldots q_{N^{\prime}}^{\prime} \rangle\left\langle q_{1} \ldots q_{N}\left|c_{q}^{\dagger}\right| q_{1}^{\prime} \ldots q_{N^{\prime}}^{\prime}\right\rangle^{*}
\end{aligned}
\end{equation}

Creation operator and annihilation operator are related by anticommutation rule:
\begin{equation}
\begin{aligned}
	&\left(c_{p} c_{q}^{\dagger}+c_{q}^{\dagger} c_{p}\right) | q_{1} \ldots q_{N} \rangle
	\\
	=&\delta_{p q} | q_{1} \ldots q_{N} \rangle-\delta_{p, q_{N}} | q_{1} \ldots q_{N-1} q \rangle+\ldots
	\\
	&+ \delta_{p, q_{N}} | q_{1} \ldots q_{N-1} q \rangle-\delta_{p, q_{N-1}} | q_{1} \ldots q_{N-2} q_{N} q \rangle+\ldots
	\\
	=&\delta_{p q} | q_{1} \ldots q_{N} \rangle
\end{aligned}
\end{equation}

Thus 
\begin{equation}
\left\{c_{q}^{\dagger}, c_{p}\right\}=c_{p} c_{q}^{\dagger}+c_{q}^{\dagger} c_{p}=\delta_{p q}
\end{equation}

In the similar way,
\begin{equation}
\left\{c_{p}^{\dagger}, c_{q}^{\dagger}\right\}=0, \quad\left\{c_{p}, c_{q}\right\}=0
\end{equation}

In fact, the anticommutation rule comes from the fermion attribute of electron.
For bosons, the anticommutation should be replaced by commutation rule.

The advantage of creation and annihilation operator is obvious: we don't need to explictly antisymmetrize the N-body quantum state, which has a rather complicated expression.
In fact, we change from the common coordinate or momentum representation to particle number representation.
Since particles are identicle for either fermions and bosons, we can only tell the number of particles on each state instead of the state of each particle.
Thus, particle number representation and introduction of creation and annihilation operator remove redundent information and generate the anticommutation rule for N-body states automatically.

The introduction of creation and annihilation operators can not only help us with description of quantum states but also with operators.

Generally, a one-particle operartor can be written as
\begin{equation}
\hat{W}^{\prime}=\sum_{p, q} w_{p q} c_{p}^{\dagger} c_{q}
\end{equation}
which no longer has a summation over all the particles like the case of quantum mechanics:
\begin{equation}
\hat{W}=\sum_{i=1}^{N} \hat{w}(i)
\end{equation}

Its expectation value on a N-body quantum state is:
\begin{equation}
\left\langle q_{1} \ldots q_{N}\left|\hat{W}^{\prime}\right| q_{1} \ldots q_{N}\right\rangle=\sum_{i=1}^{N} w_{q_{i} q_{i}}
\end{equation}
which is as expected

Take the kinetic energy operator as example, since all the electrons are identical, the operator that acts on each single electron must be the same.
Thus, the quantum mechanics approach of operator cause redundent information, which is removed by second quantization.

Another important example of one-body operator is particle number operator
\begin{equation}
\hat{N}=\sum_{p} c_{p}^{\dagger} c_{p}, \quad \hat{N} | q_{1} \ldots q_{N} \rangle=N | q_{1} \ldots q_{N} \rangle
\end{equation}


The second quantization of two-body operator
\begin{equation}
\hat{V}=\sum_{i<j=1}^{N} \hat{v}(i, j)
\end{equation}
is
\begin{equation}
\hat{V}=\frac{1}{2} \sum_{p, q, r, s} V_{p q r s} c_{p}^{\dagger} c_{q}^{\dagger} c_{s} c_{r}
\end{equation}

Its expectation value is
\begin{equation}
\left\langle q_{1} \ldots q_{N}|\hat{V}| q_{1} \ldots q_{N}\right\rangle=\frac{1}{2} \sum_{q_{i}<q_{j}}\left(V_{q_{i} q_{j}\left[q_{i} q_{j}\right]}-V_{q_{j} q_{i}\left[q_{i} q_{j}\right]}\right)=\sum_{q_{i}<q_{j}} V_{q_{i} q_{j}\left[q_{i} q_{j}\right]}
\end{equation}

The change of basis will transform creation and annihilation operator just like normal operators:
\begin{equation}
\begin{aligned} 
	b_{s}^{\dagger} | q_{1} \ldots q_{N} \rangle&= | q_{1} \ldots q_{N} \tilde{s} \rangle
	\\
	&= \sum_{q} | q_{1} \ldots q_{N} q \rangle\langle q | \tilde{s}\rangle
	\\
	&=\sum_{q}\langle q | \tilde{s}\rangle c_{q}^{\dagger} | q_{1} \ldots q_{N} \rangle
\end{aligned}
\end{equation}
and thus
\begin{equation}
\begin{aligned}
	b_{s}^{\dagger}&=\sum_{q}\langle q | \tilde{s}\rangle c_{q}^{\dagger}
	\\
	b_{s}&=\sum_{q}\langle\tilde{s} | q\rangle c_{q}
\end{aligned}
\end{equation}

If the basis $\ket{q}$ contains both spacial part and spin part with spacical part $x$ and spin part $\sigma$, then it's convenient to define field operators:
\begin{equation}
\begin{aligned} \hat{\psi}^{\dagger}(\xi) &=\sum_{q} c_{q}^{\dagger} \psi_{q}^{*}(\xi) \\ \hat{\psi}(\xi) &=\sum_{q} c_{q} \psi_{q}(\xi) \end{aligned}
\end{equation}
where $\xi$ is a set of spacial and spin index.

They also have anticommutation rules like creation and annihilation operators:
\begin{equation}
\begin{array}{c}{\left\{\hat{\psi}^{\dagger}(\xi), \hat{\psi}\left(\xi^{\prime}\right)\right\}=\delta\left(x-x^{\prime}\right) \delta_{\sigma \sigma^{\prime}}} \\ {\left\{\hat{\psi}(\xi), \hat{\psi}\left(\xi^{\prime}\right)\right\}=0, \quad\left\{\hat{\psi}^{\dagger}(\xi), \hat{\psi}^{\dagger}\left(\xi^{\prime}\right)\right\}=0}\end{array}
\end{equation}

In addition, normal operators mentioned early can also be written in the form of field operators:
\begin{equation}
\begin{aligned} \hat{W} &=\int \mathrm{d} \xi \hat{w}(\xi) \hat{\psi}^{\dagger}(\xi) \hat{\psi}(\xi) \\ \hat{V} &=\frac{1}{2} \iint \mathrm{d} \xi \mathrm{d} \xi^{\prime} \hat{v}\left(\xi, \xi^{\prime}\right) \hat{\psi}^{\dagger}(\xi) \hat{\psi}^{\dagger}\left(\xi^{\prime}\right) \hat{\psi}\left(\xi^{\prime}\right) \hat{\psi}(\xi) \end{aligned}
\end{equation}

\subsection{Green Function} \label{mathrefs}

In mathematics, a Green's function of an inhomogeneous linear differential operator defined on a domain with specified initial conditions or boundary conditions is its impulse response.
Green function is a important concept in physics, since the expression of green function can easily give the result for an arbitary source or boundary condition.

Take electromagetic wave equation of electric potential in classical field theory as an example:
\begin{equation}
\left(\nabla^{2}-\frac{1}{c^2}\frac{\partial^{2}}{\partial t^{2}}\right) \phi(\mathbf{r},t)=-\rho(\mathbf{r},t)
\end{equation}
which give the following solution:
\begin{equation}
\phi(\mathbf{r}, t)=\int \frac{\delta\left(t^{\prime}+\frac{\left|\mathbf{r}-\mathbf{r}^{\prime}\right|}{c}-t\right)}{\left|\mathbf{r}-\mathbf{r}^{\prime}\right|} \rho\left(\mathbf{r}^{\prime}, t^{\prime}\right) d^{3} r^{\prime} d t^{\prime}
\end{equation}

It's easy to find that $\phi$ is propotional to $\rho$, thus if we define Green function as
\begin{equation}
\left(\nabla^{2}-\frac{1}{c^2}\frac{\partial^{2}}{\partial t^{2}}\right) G(\mathbf{r},t;\mathbf{r^{\prime}},t^{\prime})=-\delta(\mathbf{r}-\mathbf{r^\prime})\delta(t-t^{\prime})
\end{equation}

We should have
\begin{equation}
G(\mathbf{r},t;\mathbf{r^{\prime}},t^{\prime})=
-\frac{\delta\left(t^{\prime}+\frac{\left|\mathbf{r}-\mathbf{r}^{\prime}\right|}{c}-t\right)}{\left|\mathbf{r}-\mathbf{r}^{\prime}\right|}
\end{equation}
and
\begin{equation}
\phi(\mathbf{r}, t)=
\int G(\mathbf{r},t;\mathbf{r^{\prime}},t^{\prime}) \rho\left(\mathbf{r}^{\prime}, t^{\prime}\right) d^{3} r^{\prime} d t^{\prime}
\end{equation}

In many-body field theory, Green function also acts similar with that of classical field theory in the above example.
While the formal definition looks rather abstract and even forbidding, the benefits afforded by an approach based on the electron propagator should become clear after the theory has been more fully described.

The Hamiltonian in second quantization form is:
\begin{equation}
\hat{H}=\hat{T}+\hat{V}=\sum t_{p q} c_{p}^{\dagger} c_{q}+\frac{1}{2} \sum V_{p q r s} c_{p}^{\dagger} c_{q}^{\dagger} c_{s} c_{r}
\end{equation}

In many-body field theory, Heisenberg picture is the default picture used.
Thus, the time dependence of operator should be
\begin{equation}
O[t]=e^{i \hat{H} (t-t_0)} O[t_0] e^{-i \hat{H} (t-t_0)}
\end{equation}
which also holds for creation and annihilation operators
\begin{equation}
c_{p}^{\dagger}[t]=e^{i \hat{H} t} c_{p}^{\dagger} e^{-i \hat{H} t}, \quad c_{p}[t]=e^{i \hat{H} t} c_{p} e^{-i \hat{H} t}
\end{equation}

The Green function or electron propagator is defined as
\begin{equation} \label{eq:greendef}
G_{p q}\left(t, t^{\prime}\right)=-i \theta\left(t-t^{\prime}\right)\left\langle\Psi_{0}\left|c_{p}[t] c_{q}^{\dagger}\left[t^{\prime}\right]\right| \Psi_{0}\right\rangle+ i \theta\left(t^{\prime}-t\right)\left\langle\Psi_{0}\left|c_{q}^{\dagger}\left[t^{\prime}\right] c_{p}[t]\right| \Psi_{0}\right\rangle
\end{equation}
where $\theta(t)$ is step function defined as
\begin{equation}
\theta(t)=\left\{\begin{array}{ll}{1,} & {t>0} \\ {0,} & {t<0}\end{array}\right.
\end{equation}

By defining time ordering operator $\hat{\mathcal{T}}$ as
\begin{equation}
\hat{\mathcal{T}}\left(c_{p}[t] c_{q}^{\dagger}\left[t^{\prime}\right]\right)=\left\{\begin{array}{cc}{c_{p}[t] c_{q}^{\dagger}\left[t^{\prime}\right],} & {t>t^{\prime}} \\ {-c_{q}^{\dagger}\left[t^{\prime}\right] c_{p}[t],} & {t<t^{\prime}}\end{array}\right.
\end{equation}
which put operators with larger times to the left of those with smaller times,
we can write the Green function as
\begin{equation}
G_{p q}\left(t, t^{\prime}\right)=-i\left\langle\Psi_{0}\left|\hat{\mathcal{T}}\left(c_{p}[t] c_{q}^{\dagger}\left[t^{\prime}\right]\right)\right| \Psi_{0}\right\rangle
\end{equation}

The time ordering operator is necessary since operators with smaller times should be applied to states earlier than those with larger times.

If creation operator is applied before annihilation operator, a middle state with $N+1$ electrons will be produced, otherwise a middle state with $N-1$ electrons will be produced.

Thus It's convenient to divide the Green function to two parts:
\begin{equation}
\boldsymbol{G}\left(t, t^{\prime}\right)=\boldsymbol{G}^{+}\left(t, t^{\prime}\right)+\boldsymbol{G}^{-}\left(t, t^{\prime}\right)
\end{equation}
where $G^{+}$ represents the $N+1$ part and $G^{-}$ represents the $N-1$ part.

In detail,
\begin{equation}
\begin{aligned}
	G_{p q}^{+}\left(t, t^{\prime}\right) 
	&=-i \theta\left(t-t^{\prime}\right)\left\langle\Psi_{0}\left|e^{i \hat{H} t} c_{p} e^{-i \hat{H} t} e^{i \hat{H} t^{\prime}} c_{q}^{\dagger} e^{-i \hat{H} t^{\prime}}\right| \Psi_{0}\right\rangle
	\\
	&=- i \theta\left(t-t^{\prime}\right) e^{i E_{0}\left(t-t^{\prime}\right)}\left\langle\Psi_{0}\left|c_{p} e^{-i \hat{H}(t-t)^{\prime}} c_{q}^{\dagger}\right| \Psi_{0}\right\rangle
	\\
	&=-i \theta\left(t-t^{\prime}\right) \sum_{n} e^{-i\left(E_{n}^{N+1}-E_{0}\right)\left(t-t^{\prime}\right)}\left\langle\Psi_{0}\left|c_{p}\right| \Psi_{n}^{N+1}\right\rangle\left\langle\Psi_{n}^{N+1}\left|c_{q}^{\dagger}\right| \Psi_{0}\right\rangle
\end{aligned}
\end{equation}
where $\ket{\Psi_n^{N+1}}$ is the orthonomal set of eigenstates of Hamiltonian with $N+1$ electrons.

It is similar for the $G^{-}$ part:
\begin{equation}
G_{p q}^{-}\left(t, t^{\prime}\right)=i \theta\left(t^{\prime}-t\right) \sum_{n} e^{i\left(E_{n}^{N-1}-E_{0}\right)\left(t-t^{\prime}\right)}\left\langle\Psi_{0}\left|c_{q}^{\dagger}\right| \Psi_{n}^{N-1}\right\rangle\left\langle\Psi_{n}^{N-1}\left|c_{p}\right| \Psi_{0}\right\rangle
\end{equation}

Note that the ionization potentials and electron affinities have appeared, which are exactly what we need:
\begin{equation}
\begin{array}{l}{A_{n}=E_{0}-E_{n}^{N+1}} \\ {I_{n}=E_{n}^{N-1}-E_{0}}\end{array}
\end{equation}

Until now, we are working on time representation, i.e. the parameters of Green function is time.
However, it is useful to switch to energy(frequency) representation by fourier transformation:
\begin{equation}
G_{p q}(\omega)=\int_{-\infty}^{\infty} e^{i \omega\left(t-t^{\prime}\right)} G_{p q}\left(t, t^{\prime}\right) \mathrm{d}\left(t-t^{\prime}\right)
\end{equation}

However, there will be a problem of singularity during the integration.
Take $G^{+}$ as example, the integration over time-dependent part of $G^{+}$ is
\begin{equation}
\begin{aligned} f_{n}^{+}(\omega) &=\int_{-\infty}^{\infty} e^{i \omega \tau}\left[-i \theta(\tau) e^{-i\left(E_{n}^{N+1}-E_{0}\right) \tau}\right] \mathrm{d} \tau \\ &=-i \int_{0}^{\infty} e^{i\left[\omega-E_{n}^{N+1}+E_{0}\right] \tau} \mathrm{d} \tau \end{aligned}
\end{equation}
which is ill-behaved at the upper limit of time.

A common mathematical trick to solve this singularity problem is to add a infinitely small negative part to the exponent:
\begin{equation}
e^{i\left[\omega-E_{n}^{N+1}+E_{0}\right] \tau} \rightarrow e^{i\left[\omega-E_{n}^{N+1}+E_{0}+i\eta\right] \tau}
\end{equation}
which will give the result
\begin{equation}
f_{n}^{+}(\omega)=\frac{1}{\omega-E_{n}^{N+1}+E_{0}+i \eta}
\end{equation}

This trick works because all what we deal with now are intermediate results and are not any kind of observable.
When we get to observable finally, it will be the time to let $\eta \rightarrow 0$ and will give normal physical quantities.

On the other hand, when we perform inverse fourier transformation to get Green function back to time representation, residue theorem is used and the sign of $\eta$ will determine the direction of contour integration and thus the sign of time difference.

Similarily, we can also apply the same trick to the $G^{-}$ part.
By including the time-independent part, we obtain the Green function in energy representation:
\begin{equation} \label{spectralrepresentation}
G_{p q}(\omega)=\sum_{n} \frac{\left\langle\Psi_{0}\left|c_{p}\right| \Psi_{n}^{N+1}\right\rangle\left\langle\Psi_{n}^{N+1}\left|c_{q}^{\dagger}\right| \Psi_{0}\right\rangle}{\omega+E_{0}-E_{n}^{N+1}+i \eta}+\sum_{n} \frac{\left\langle\Psi_{0}\left|c_{q}^{\dagger}\right| \Psi_{n}^{N-1}\right\rangle\left\langle\Psi_{n}^{N-1}\left|c_{p}\right| \Psi_{0}\right\rangle}{\omega+E_{n}^{N-1}-E_{0}-i \eta}
\end{equation}
which is also refered as spectral representation or Lehmann representation.
It is worthy noticing that all the energies that we need are located on the poles in complex plane, while ionization potentials on the upper half and electron affinities on the lower half.

It is also convenient to define spectroscopic factors as:
\begin{equation}
\begin{array}{ll}{x_{p}^{(n)}=\left\langle\Psi_{0}\left|c_{p}\right| \Psi_{n}^{N+1}\right\rangle,} & { n \in\{N+1\}} \\ {x_{p}^{(n)}=\left\langle\Psi_{n}^{N-1}\left|c_{p}\right| \Psi_{0}\right\rangle,} & { n \in\{N-1\}}\end{array}
\end{equation}

The spectroscopic factor is closely related to photoioniation process.
The photoioniation cross section is
\begin{equation}
\sigma_{n}(\epsilon) \sim \frac{2}{3} \varepsilon\left|\sum_{p}\langle\varepsilon|\hat{d}| p\rangle x_{p}^{(n)}\right|^{2}
\end{equation}
where the spectroscopic factors weight the participation of individual orbitals in the final ionic state.

By using the anticommutation rule of creation and annihilation operators, the spectroscopic factors satisfy the following orthonomal relation:
\begin{equation}
\sum_{n \in\{N+1\}} x_{p}^{(n)} x_{q}^{(n) *}+\sum_{n \in\{N-1\}} x_{p}^{(n)} x_{q}^{(n) *}=\delta_{p q}
\end{equation}

The spectral representation can also be written in a more compact form as
\begin{equation}
\begin{aligned} G_{p q}^{+}(\omega) &=\left\langle\Psi_{0}\left|c_{p}\left(\omega-\hat{H}+E_{0}+i \eta\right)^{-1} c_{q}^{\dagger}\right| \Psi_{0}\right\rangle \\ G_{p q}^{-}(\omega) &=\left\langle\Psi_{0}\left|c_{q}^{\dagger}\left(\omega+\hat{H}-E_{0}-i \eta\right)^{-1} c_{p}\right| \Psi_{0}\right\rangle \end{aligned}
\end{equation}

From above discussions, we know that Green function can derive spectroscopic factors which can be used to determine photoioniation cross section.
In addition, Green function can also derive ground state density matrix and thus energy.

The ground state energy is
\begin{equation}
E_{0}=\left\langle\Psi_{0}|\hat{H}| \Psi_{0}\right\rangle=\sum t_{r s}\left\langle\Psi_{0}\left|c_{r}^{\dagger} c_{r}\right| \Psi_{0}\right\rangle+\frac{1}{2} \sum V_{r s u v}\left\langle\Psi_{0}\left|c_{r}^{\dagger} c_{s}^{\dagger} c_{v} c_{u}\right| \Psi_{0}\right\rangle
\end{equation}

Let us first consider the time derivative of the time-dependent annihilation operator:
\begin{equation}
\begin{aligned}
	i \frac{\partial}{\partial t} c_{p}[t]&=i \frac{\partial}{\partial t}\left(e^{i \hat{H} t} c_{p} e^{-i \hat{H} t}\right)=e^{i \hat{H} t}\left[c_{p}, \hat{H}\right] e^{-i \hat{H} t}
	\\
	&=\sum_{s} t_{p s} c_{s}[t]+\sum_{s, u, v} V_{p s u v} c_{s}^{\dagger}[t] c_{v}[t] c_{u}[t]
\end{aligned}
\end{equation}

Thus we can calculate the time derivative of Green function as
\begin{equation}
\begin{aligned} i \frac{\partial}{\partial t} G_{p q}\left(t, t^{\prime}\right)=& \delta\left(t-t^{\prime}\right)\left\langle\Psi_{0} \|\left\{c_{p}[t], c_{q}^{\dagger}\left[t^{\prime}\right]\right\}\right\} \Psi_{0} \\ &-i\left\langle\Psi_{0}\left|\hat{\mathcal{T}}\left[\left(i \frac{\partial}{\partial t} c_{p}[t]\right) c_{q}^{\dagger}\left[t^{\prime}\right]\right]\right| \Psi_{0}\right\rangle \\=& \delta_{p q} \delta\left(t-t^{\prime}\right)-i \sum_{s} t_{p s}\left\langle\Psi_{0}\left|\hat{\mathcal{T}}\left[c_{s}[t] c_{q}^{\dagger}\left[t^{\prime}\right]\right]\right| \Psi_{0}\right\rangle \\ &- i \sum_{s, u, v} V_{p s u v}\left\langle\Psi_{0}\left|\hat{\mathcal{T}}\left[c_{s}^{\dagger}[t] c_{v}[t] c_{u}[t] c_{q}^{\dagger}\left[t^{\prime}\right]\right]\right| \Psi_{0}\right\rangle \end{aligned}
\end{equation}

By approaching $t^{\prime}$ to $t^{+}$ and take the trace of the equation above, we obtain
\begin{equation}
\sum_{p} i \frac{\partial}{\partial t} G_{p p}\left(t, t^{+}\right)-i\left\langle\Psi_{0}|\hat{T}| \Psi_{0}\right\rangle= 2 i\left\langle\Psi_{0}|\hat{V}| \Psi_{0}\right\rangle
\end{equation}

By using residue theorem, the contour integration of $\omega G_{pq}(\omega)$ can be calculate as
\begin{equation}
\begin{aligned}
	\frac{1}{2 \pi i} \oint \omega G_{p q}^{-}(\omega) \mathrm{d} \omega
	&=-\sum_{n}\left(E_{n}^{N-1}-E_{0}\right)\left\langle\Psi_{0}\left|c_{q}^{\dagger}\right| \Psi_{n}^{N-1}\right\rangle\left\langle\Psi_{n}^{N-1}\left|c_{p}\right| \Psi_{0}\right\rangle
\\ 
&=\left\langle\Psi_{0}\left|c_{q}^{\dagger}\left[c_{p}, \hat{H}\right]\right| \Psi_{0}\right\rangle
\\
	&=\left\langle\Psi_{0}|\hat{T}| \Psi_{0}\right\rangle+ 2\left\langle\Psi_{0}|\hat{V}| \Psi_{0}\right\rangle
\end{aligned}
\end{equation}

Thus the ground state enengy can be obtained as:
\begin{equation} \label{greenenergy}
E_{0}=\frac{1}{4 \pi i} \oint \operatorname{Tr}[(\omega \mathbf{1}+\boldsymbol{T}) \boldsymbol{G}(\omega)] \mathrm{d} \omega=\frac{1}{4 \pi i} \oint \operatorname{Tr}\left[(\omega \mathbf{1}+\boldsymbol{T}) \boldsymbol{G}^{-}(\omega)\right] \mathrm{d} \omega
\end{equation}

This contour integration can be done by residue theorem after we calculate ionization potentials and electron affinities later.

Until now, we have not given any practical approaches to calculate Green function.
We will later see that general Green function can be calculated by Feynman expansion, which is essentially a perturbation theory.
Thus we need to first solve the unperturbed Green function, which is usually refered to as free Green function.
\begin{equation}
\hat{H}_{0}=\sum \varepsilon_{r} c_{r}^{\dagger} c_{r}
\end{equation}

Then we calculate the time dependence of creation and annihilation operators:
\begin{equation}
i \frac{\partial}{\partial t} c_{p}(t)=e^{i \hat{H}_{0} t}\left[c_{p}, \hat{H}_{0}\right] e^{-i \hat{H}_{0} t}=\varepsilon_{p} c_{p}(t)
\end{equation}
thus
\begin{equation}
c_{p}(t)=e^{i \hat{H}_{0} t} c_{p} e^{-i \hat{H}_{0} t}=e^{-i \varepsilon_{p} t} c_{p}
\end{equation}

From the definition of Green function in Eq \ref{eq:greendef}, we have 
\begin{equation}
G_{p q}^{0}\left(t, t^{\prime}\right)=-i e^{-i \varepsilon_{p}\left(t-t^{\prime}\right)} \delta_{p q}\left(\theta\left(t-t^{\prime}\right) \overline{n}_{p}-\theta\left(t^{\prime}-t\right) n_{p}\right)
\end{equation}
where
\begin{equation}
n_{p}=
\left\{
\begin{array}{ll}
	{1,} & {p \leq N}
	\\
	{0,} & {p>N}
\end{array}
\right.
\end{equation}
is occupation number of orbital $p$.

The corresponding energy representation is
\begin{equation}
G_{p q}^{0}(\omega)=\delta_{p q}\left(\frac{\overline{n}_{p}}{\omega-\varepsilon_{p}+i \eta}+\frac{n_{p}}{\omega-\varepsilon_{p}-i \eta}\right)
\end{equation}

\subsection{Perturbation Theory for Green Function}

We discussed teh definition of Green function and how to derive important physical quantities from Green function in the previously part.
In this part, we will give a perturbation scheme to expand Green function into different orders.

We start from the introduction of M{\o}ller-Plesset partitioning:
\begin{equation}
\hat{H}=\hat{T}+\hat{V}=\hat{H}_{0}+\hat{H}_{I}
\end{equation}
where
\begin{equation}
\hat{H}_{0}=\sum \epsilon_{r} c_{r}^{\dagger} c_{r}
\end{equation}
is the Hartree-Fock Hamiltonian calculated from the solutions of Hartree-Fock equation in second quantization form:
\begin{equation}
t_{r s}+\sum_{k} V_{r k[s k]} n_{k}=\epsilon_{r} \delta_{r s}
\end{equation}
which gives the form of $\hat{H}_I$:
\begin{equation}
\hat{H}_I=\frac{1}{2} \sum V_{p q r s} c_{p}^{\dagger} c_{q}^{\dagger} c_{s} c_{r}-\sum_{k} V_{r k[s k]} n_{k} c_{r}^{\dagger} c_{s}
\end{equation}

For Hartree-Fock ground state:
\begin{equation}
\hat{H}_{0} | \Phi_{0} \rangle=E_{0}^{(0)} | \Phi_{0} \rangle
\end{equation}

Before working on the perturbation theory, we need to transfer to interaction picture.
The basic idea of interaction picture is to remove the rapidly fluctuating exponetial factor caused by the trivial unperturbed Hamiltonian before the quantum state:
\begin{equation}
i \frac{\partial}{\partial t} | \Psi(t) \rangle=\hat{H}_{0}+\hat{H}_{1} | \Psi(t) \rangle
\end{equation}

By perturbation assumption, $\hat{H}_1$ is small compared with $\hat{H}_0$, thus the fluctuating factor $e^{-i \hat{H}_0 t}$ caused by unperturbed part dominates the time dependence of $\Psi$.
Thus, the change to interaction picture will help us to focus on the perturbed part of Hamiltonian.

In interaction picture, the quantum state is defined as
\begin{equation}
| \Psi_{I}(t) \rangle=e^{i \hat{H}_{0} t} | \Psi_{S}(t) \rangle
\end{equation}
and thus the Schrodinger equation becomes:
\begin{equation}
i \frac{\partial}{\partial t} | \Psi_{I}(t) \rangle=\hat{H}_{I}(t) | \Psi_{I}(t) \rangle
\end{equation}
where the $\hat{H}_I$ is the perturbation part of Hamiltonian in interaction picture:
\begin{equation}
\hat{H}_{I}(t)=e^{i \hat{H}_{0} t} \hat{H}_{1}(t) e^{-i \hat{H}_{0} t}
\end{equation}

As is expected, the unperturbed part does not exist any more, thus the change of quantum state is slow under the small perturbation assumption.
In fact, the time evolution of quantum state in interaction picture is:
\begin{equation}
| \Psi_{I}(t) \rangle=\hat{U}\left(t, t_{0}\right) | \Psi_{I}\left(t_{0}\right) \rangle
\end{equation}
where
\begin{equation}
\hat{U}\left(t, t_{0}\right)=e^{i \hat{H}_{0} t} e^{-i \hat{H}\left(t-t_{0}\right)} e^{-i \hat{H}_{0} t_{0}}
\end{equation}
is the time evolution opertator in interction picture.

Equation of motion of the evolution operator itself is:
\begin{equation}
i \frac{\partial}{\partial t} \hat{U}\left(t, t_{0}\right)=\hat{H}_{I}(t) U\left(t, t_{0}\right)
\end{equation}

Thus it is possible to expand the evolution operator in series:
\begin{equation}
\hat{U}\left(t, t_{0}\right)=\hat{\mathbb{1}}-i \int_{t_{0}}^{t} \mathrm{d} t_{1} \hat{H}_{I}\left(t_{1}\right)+(-i)^{2} \int_{t_{0}}^{t} \mathrm{dt}_{1} \int_{t_{0}}^{t_{1}} \mathrm{d} t_{2} \hat{H}_{I}\left(t_{1}\right) \hat{H}_{I}\left(t_{2}\right)+\ldots
\end{equation}
and
\begin{equation}
\hat{U}^{(n)}\left(t, t_{0}\right)=
(-i)^{n} \int_{t_n<t_{n-1}<\dots<t_1<t_0<t} \mathrm{d} t_{1} \mathrm{d} t_{2} \ldots \mathrm{d} t_{n} \hat{H}_{I}\left(t_{1}\right) \hat{H}_{I}\left(t_{2}\right) \ldots \hat{H}_{I}\left(t_{n}\right)
\end{equation}
is the general expression for the $n$th order.

Note that the perturbation part of Hamiltonians in interction picture do not commute for different time arguments, i.e.
\begin{equation}
\left[\hat{H}_{I}(t), \hat{H}_{I}\left(t^{\prime}\right)\right] \neq 0 \text { for } t \neq t^{\prime}
\end{equation}

However, the time arguments themselves can be exchanged since they are intermediate variables.
Thus,
\begin{equation}
\begin{aligned}
	\int_{t_n<t_{n-1}<\dots<t_1<t_0<t} &\mathrm{d} t_{1} \mathrm{d} t_{2} \ldots \mathrm{d} t_{n} \hat{H}_{I}\left(t_{1}\right) \hat{H}_{I}\left(t_{2}\right) \ldots \hat{H}_{I}\left(t_{n}\right)
	\\
	=\int_{t_{P_n}<t_{P_{n-1}}<\dots<t_{P_1}<t_{P_0}<t} &\mathrm{d} t_{1} \mathrm{d} t_{2} \ldots \mathrm{d} t_{n} \hat{H}_{I}\left(t_{P_1}\right) \hat{H}_{I}\left(t_{P_2}\right) \ldots \hat{H}_{I}\left(t_{P_n}\right)
\end{aligned}
\end{equation}
where $P$ is any possible permuation.

Thus each order of evolution operator can be written in the form of time ordering operator:
\begin{equation}
\hat{U}^{(n)}\left(t, t_{0}\right)=\frac{(-i)^{n}}{n !} \int_{t_{0}}^{t} \mathrm{d} t_{1} \ldots \int_{t_{0}}^{t} \mathrm{d} t_{n} \hat{\boldsymbol{T}}\left[\hat{H}_{I}\left(t_{1}\right) \ldots \hat{H}_{I}\left(t_{n}\right)\right]
\end{equation}

In a symbolic way, the evolution operator can be written as
\begin{equation}
\hat{U}\left(t, t_{0}\right)=\hat{\mathcal{T}} e^{-i \int_{t_{0}}^{t} \mathrm{d} t^{\prime} \hat{H}_{I}\left(t^{\prime}\right)}
\end{equation}

Think about that at time $t=-\infty$, the Hamiltonian of a system is $\hat{H}_0$ and the system is in the ground state.
Then the perturbed Hamiltonian $\hat{H}_1$ is gradually turned on exponetially, and at time $t=0$, the perturbed Hamiltonian is fully turned on:
\begin{equation}
\hat{H}(t)=\hat{H}_{0}+e^{-\epsilon|t|} \hat{H}_{I}
\end{equation}
In the limit of $\epsilon \rightarrow 0$, which is refered to as adiabatic limit, it is expected that the system in the ground state of the total Hamiltonian at $t=0$.

The time evolution operator associated with this process is
\begin{equation}
\hat{U}_{\epsilon}\left(t, t_{0}\right)=\sum_{n=0}^{\infty} \frac{(-i)^{n}}{n !} \int_{t_{0}}^{t} \mathrm{d} t_{1} e^{-\epsilon\left|t_{1}\right|} \ldots \int_{t_{0}}^{t} \mathrm{d} t_{n} e^{-\epsilon\left|t_{n}\right|} \hat{\mathcal{T}}\left[\hat{H}_{I}\left(t_{1}\right) \ldots \hat{H}_{I}\left(t_{n}\right)\right]
\end{equation}

However, the situation is not that simple and the evolution operator actually does not converge when $\epsilon \rightarrow 0$.
The divergent is caused by the phase factor of the quantum state, which is not physical.
Thus, it is possible to cancel this unphysical divergence, which leads to the famous Gell-Mann and Low theorem:
\begin{equation}
| \Psi_{0}^{\prime} \rangle=\lim _{\epsilon \rightarrow 0} \frac{\hat{U}_{\epsilon}(0,-\infty) | \Phi_{0} \rangle}{\left\langle\Phi_{0}\left|\hat{U}_{\epsilon}(0,-\infty)\right| \Phi_{0}\right\rangle}
\end{equation}
is the ground state of $\hat{H}$ with the eigenvalue
\begin{equation}
E_{0}=E_{0}^{(0)}+\lim _{\epsilon \rightarrow 0} \frac{\left\langle\Phi_{0}\left|\hat{H}_{I} \hat{U}_{\epsilon}(0,-\infty)\right| \Phi_{0}\right\rangle}{\left\langle\Phi_{0}\left|\hat{U}_{\epsilon}(0,-\infty)\right| \Phi_{0}\right\rangle}
\end{equation}
and the proof is given in the refernce \cite{gallmannlow}

Since both the numerator and denominator diverges with $\epsilon \rightarrow 0$, the convergence holds only when one calculate contribution from each order separately and then add them up.
In addition, the normalization rule for the final state $| \Psi_{0}^{\prime}\rangle$ is 
\begin{equation}
\left\langle\Phi_{0} | \Psi_{0}^{\prime}\right\rangle= 1
\end{equation}
rather than
\begin{equation}
\left\langle\Psi_{0}^{\prime} | \Psi_{0}^{\prime}\right\rangle= 1
\end{equation}

And it is also similar for the $t>0$ scheme:
\begin{equation}
| \Psi_{0}^{\prime \prime} \rangle=\lim _{\epsilon \rightarrow 0} \frac{\hat{U}_{\epsilon}(0, \infty) | \Phi_{0} \rangle}{\left\langle\Phi_{0}\left|\hat{U}_{\epsilon}(0, \infty)\right| \Phi_{0}\right\rangle}=\lim _{\epsilon \rightarrow 0} \frac{\hat{U}_{\epsilon}(0,-\infty) | \Phi_{0} \rangle}{\left\langle\Phi_{0}\left|\hat{U}_{\epsilon}(0,-\infty)\right| \Phi_{0}\right\rangle}=| \Psi_{0}^{\prime} \rangle
\end{equation}

Then we will calculate the expectation value of any operator $\hat{O}_H$:
\begin{equation}
\begin{aligned}
	\left\langle\Psi_{0}\left|\hat{O}_{H}(t)\right| \Psi_{0}\right\rangle 
	&=\frac{\left\langle\Psi_{0}^{\prime}\left|\hat{O}_{H}(t)\right| \Psi_{0}^{\prime}\right\rangle}
	{\left\langle\Psi_{0}^{\prime} | \Psi_{0}^{\prime}\right\rangle}
	\\
	&=\lim _{\epsilon \rightarrow 0} 
	\frac{\left\langle\Phi_{0}\left|\hat{U}_{\epsilon}(\infty, 0)\hat{U}_{\epsilon}(0, t) \hat{O}_{I}(t) \hat{U}_{\epsilon}(0, t)\hat{U}_{\epsilon}(t,-\infty)\right| \Phi_{0}\right\rangle}
	{\left\langle\Phi_{0}\left|\hat{U}_{\epsilon}(\infty,-\infty)\right| \Phi_{0}\right\rangle}
	\\
	&=\lim _{\epsilon \rightarrow 0} 
	\frac{\left\langle\Phi_{0}\left|\hat{U}_{\epsilon}(\infty, t) \hat{O}_{I}(t) \hat{U}_{\epsilon}(t,-\infty)\right| \Phi_{0}\right\rangle}
	{\left\langle\Phi_{0}\left|\hat{U}_{\epsilon}(\infty,-\infty)\right| \Phi_{0}\right\rangle}
	\\
	&=\lim _{\epsilon \rightarrow 0} \sum_{n=0}^{\infty} \frac{(-i)^{n}}{n !} \int_{-\infty}^{\infty} \mathrm{d} t_{1} e^{-\epsilon\left|t_{1}\right|} \ldots \int_{-\infty}^{\infty} \mathrm{d} t_{n} e^{-\epsilon\left|t_{n}\right|}
	\\
	&\qquad \frac{\left\langle\Phi_{0}\left|
	\hat{\mathcal{T}}\left[\hat{H}_{I}\left(t_{1}\right) \ldots \hat{H}_{I}\left(t_{n}\right) \hat{O}_{I}(t)\right]\right|
	\Phi_{0}\right\rangle}
	{\left\langle\Phi_{0}\left|\hat{U}_{\epsilon}(\infty,-\infty)\right| \Phi_{0}\right\rangle}
\end{aligned}
\end{equation}

Similarly, for expectation of two opeartors with different time, we have
\begin{equation}
\begin{aligned}
	\left\langle\Psi_{0}\left|\hat{\mathcal{T}}\left[\hat{P}_{H}(t) \hat{Q}_{H}\left(t^{\prime}\right)\right]\right| \Psi_{0}\right\rangle
	&=\lim _{\epsilon \rightarrow 0} \sum_{n=0}^{\infty} \frac{(-i)^{n}}{n !} \int_{-\infty}^{\infty} \mathrm{d} t_{1} e^{-\epsilon\left|t_{1}\right|} \ldots \int_{-\infty}^{\infty} \mathrm{d} t_{n} e^{-\epsilon\left|t_{n}\right|}
	\\
	&\qquad =\frac{\left\langle\Phi_{0}\left|
	\hat{\boldsymbol{T}}\left[\hat{H}_{I}\left(t_{1}\right) \ldots \hat{H}_{I}\left(t_{n}\right) \hat{P}_{I}(t) \hat{Q}_{I}\left(t^{\prime}\right)\right]\right|
	\Phi_{0}\right\rangle}
	{\left\langle\Phi_{0}\left|\hat{U}_{\epsilon}(\infty,-\infty)\right| \Phi_{0}\right\rangle}
\end{aligned}
\end{equation}

Thus, it is possible for us to expand Green function to series, which is the original purpose:
\begin{equation} \label{greenexp}
\begin{aligned}
	i G_{p q}\left(t, t^{\prime}\right)
	=&\left\langle\Psi_{0}\left|
	\hat{\mathcal{T}}\left[c_{p}[t] c_{q}^{\dagger}\left[t^{\prime}\right]\right]\right|
	\Psi_{0}\right\rangle
	\\
	=& \lim _{\epsilon \rightarrow 0} \sum_{n=0}^{\infty} \frac{(-i)^{n}}{n !} 
	\int_{-\infty}^{\infty} \mathrm{d} t_{1} e^{-\epsilon\left|t_{1}\right|} \ldots \int_{-\infty}^{\infty} \mathrm{d} t_{n} e^{-\epsilon\left|t_{n}\right|}
	\\
	& \frac{\left\langle\Phi_{0}\left|
	\hat{\boldsymbol{T}}\left[\hat{H}_{I}\left(t_{1}\right) \ldots \hat{H}_{I}\left(t_{n}\right) c_{p}(t) c_{q}^{\dagger}\left(t^{\prime}\right)\right]\right|
	\Phi_{0}\right\rangle}
	{\left\langle\Phi_{0}\left|
	\hat{U}_{\epsilon}(\infty,-\infty)\right|
	\Phi_{0}\right\rangle}
\end{aligned}
\end{equation}
and the denominator as
\begin{equation}
\begin{aligned}\left\langle\Phi_{0}\left|\hat{U}_{\epsilon}(\infty,-\infty)\right| \Phi_{0}\right\rangle=\sum_{n=0}^{\infty} \frac{(-i)^{n}}{n !} \int_{-\infty}^{\infty} \mathrm{d} t_{1} e^{-\epsilon\left|t_{1}\right|} \ldots \int_{-\infty}^{\infty} \mathrm{d} t_{n} e^{-\epsilon\left|t_{n}\right|} \\\left\langle\Phi_{0}\left|\hat{\mathcal{T}}\left[\hat{H}_{I}\left(t_{1}\right) \ldots \hat{H}_{I}\left(t_{n}\right)\right]\right| \Phi_{0}\right\rangle \end{aligned}
\end{equation}

As we have seen, field theory approach chooses a totally different way of perturbation expansion, which is using Gell-Mann and Low theorem.
However, the result of perturbation has particular physical meaning, which is the asympotic behaviour of eigenvalues and eigenstates when magnitude of perturbed Hamiltonian changes.
Thus, all perturbation theory should give the same result.

Generally, M{\o}ller-Plesset perturbation theory gives
\begin{equation} \label{rspt}
| \Psi_{0} \rangle=| \Phi_{0} \rangle+\sum_{n=1}^{\infty}\left[\frac{\hat{Q}_{0}}{E_{0}^{(0)}-\hat{H}_{0}}\left(E_{0}^{(0)}-E_{0}+\hat{H}_{I}\right)\right]^{n} | \Phi_{0} \rangle
\end{equation}
which is proved in reference \cite{rsptproof}.

In fact, we can also write the result from field theory approach into similar expression.
The numerator is:
\begin{equation} \label{expansion}
\begin{array}{l}{\hat{U}_{\epsilon}(0,-\infty) | \Phi_{0} \rangle=} \\ { | \Phi_{0} \rangle+\sum_{n=1}^{\infty}(-i)^{n} \int_{-\infty}^{0} \mathrm{d} t_{1} \int_{-\infty}^{t_{1}} \mathrm{d} t_{2} \ldots \int_{-\infty}^{t_{n-1}} \mathrm{d} t_{n} e^{\epsilon\left(t_{1}+\cdots+t_{n}\right)} \hat{H}_{I}\left(t_{1}\right) \ldots \hat{H}_{I}\left(t_{n}\right) | \Phi_{0} \rangle}\end{array}
\end{equation}

Since
\begin{equation} \label{I2H}
\hat{H}_{I}\left(t_{j}\right)=e^{i \hat{H}_{0} t_{j}} \hat{H}_{I} e^{-i \hat{H}_{0} t_{j}}
\end{equation}
and 
\begin{equation}
\hat{H}_{I}\left(t_{j}\right) \hat{H}_{I}\left(t_{j^{\prime}}\right)=e^{i \hat{H}_{0} t_{j}} \hat{H}_{I} e^{-i \hat{H}_{0}\left(t_{j}-t_{j}\right)} \hat{H}_{I} e^{-i \hat{H}_{0} t_{j^{\prime}}}
\end{equation}
, we can expect that there are a lot of terms like $e^{-i \hat{H}_{0}\left(t_{j}-t_{j}\right)}$ after we write every $\hat{H}_I(t_i)$ in Eq \ref{expansion} in terms of Eq \ref{I2H}.

Thus, we change integration variable from $t$ to the following $x$:
\begin{equation}
	\begin{aligned} x_{1} &=t_{1} \\ x_{2} &=t_{2}-t_{1} \\ x_{3} &=t_{3}-t_{2} \\ & \vdots \\ x_{n} &=t_{n}-t_{n-1} \end{aligned}
\end{equation}

Thus
\begin{equation}
	\begin{aligned}
		\hat{U}_{\epsilon}^{(n)}(0,-\infty) | \Phi_{0} \rangle&=(-i)^{n} \int_{-\infty}^{0} \mathrm{d} x_{1} e^{n \epsilon x_{1}} e^{i\left(\hat{H}_{0}-E_{0}^{(0)}\right) x_{1}} \hat{H}_{I}
		\\ 
		&\qquad {\int_{-\infty}^{0} \mathrm{d} x_{2} e^{(n-1) \epsilon x_{2}} e^{i\left(\hat{H}_{0}-E_{0}^{(0)}\right) x_{2}} \hat{H}_{I} \ldots \int_{-\infty}^{0} \mathrm{d} x_{n} e^{\epsilon x_{n}} e^{i\left(\hat{H}_{0}-E_{0}^{(0)}\right) x_{n}} \hat{H}_{I} | \Phi_{0} \rangle}
		\\
		&={\frac{1}{E_{0}^{(0)}-\hat{H}_{0}+n i \epsilon} \hat{H}_{I}}
		{\frac{1}{E_{0}^{(0)}-\hat{H}_{0}+(n-1) i \epsilon} \hat{H}_{I}} \cdots
		\\
		&\qquad \frac{1}{E_{0}^{(0)}-\hat{H}_{0}+i \epsilon} \hat{H}_{I} | \Phi_{0} \rangle
	\end{aligned}
\end{equation}

However, compared with Eq \ref{rspt}, the expression here misses the important term $Q_0$, which prevent the divergence when $\epsilon \rightarrow 0$.
This is expected, since we have mentioned that the divergent terms appearing in numerator will be canceled out in the denominator.

The denominator is:
\begin{equation}
	\left\langle\Phi_{0}\left|\hat{U}_{\epsilon}(0,-\infty)\right| \Phi_{0}\right\rangle= 1+\left\langle\Phi_{0}\left|\frac{1}{E_{0}^{(0)}-\hat{H}_{0}+i \epsilon} \hat{H}_{I}\right| \Phi_{0}\right\rangle+\cdots
\end{equation}

Although it is not easy to prove for all the orders, but it is easy to see that the first order of the total expression gives the same result with M{\o}ller-Plesset perturbation theory.



\section{Feynman Diagramm}

Having introduced Gell-Mann and Low theorem, the only left is how to calculate the expectation values of time-ordered operators in Eq \ref{greenexp}.
In this section, we will show how to use a diagrammatic approach to calculate them with the help of Wicki's theorem.

However, since we have mentioned that all perturbation theory gives the same result, one way wonder why do we need such a strange approach?
Then answer is the algebraic way to calculate perturbation terms is so complicated that it is difficult to understand what does it represent for.
In contrast, Feynman diagrams give a simple visualization of what would otherwise be an arcane and abstract formula.
As David Kaiser writes, "since the middle of the 20th century, theoretical physicists have increasingly turned to this tool to help them undertake critical calculations", and so "Feynman diagrams have revolutionized nearly every aspect of theoretical physics". \cite{kaiser}

Before introducing Wick's theorem, let's first have an observation of the expression we want to calculate:
\begin{equation}
	\left\langle\Phi_{0}\left|
	\hat{\mathcal{T}} \left[ c_u^{\dagger} c_v^{\dagger} \ldots c_i c_j \ldots \right]
	\right| \Phi_{0}\right\rangle
\end{equation}

Since creation operator add an electron while annihilation operator remove an electron, they must be paired somehow to get a nonzero final result.
Mathematically, we give the definition of physical and unphysical operators:

Since creation and annihilation operators will give the following effects on Hartree-Fock state:
\begin{equation} \label{cceffect}
	\begin{aligned}
		c_{p}^{\dagger} | \Phi_{0} \rangle=\left\{\begin{array}{ll}{ | \Phi_{p}^{N+1} \rangle} & {\text { for } n_{p}=0} \\ {0} & {\text { for } n_{p}=1}\end{array}\right.
		\\
		c_{p} | \Phi_{0} \rangle=\left\{\begin{array}{ll}{ | \Phi_{p}^{N-1} \rangle} & {\text { for } n_{p}=1} \\ {0} & {\text { for } n_{p}=0}\end{array}\right.
	\end{aligned}
\end{equation}
An operator is referred to as physical if the outcome is an $N\pm 1$ state (first and third case in \ref{cceffect}) and unphysical if the outcome is 0 (second and fourth case in \ref{cceffect}).

Then we define the normal-ordered product, which puts all physical operators to the left of unphysical operators:
\begin{equation}
	\hat{\mathcal{N}}\left[O_{i} O_{j} O_{k} \ldots\right] \equiv(-1)^{P^{\prime}} O_{P^{\prime}(i)} O_{P^{\prime}(j)} \ldots
\end{equation}

The so-called "pairing" process is formally defined as contraction:
\begin{equation}
	\wick{
		\c1 O_{r} \c1 O_{s} \equiv \hat{\mathcal{T}}\left[O_{r} O_{s}\right]-\hat{\mathcal{N}}\left[O_{r} O_{s}\right]
	}
\end{equation}

Wick's theorem \cite{wickproof} establishes a reformulation of a general time-ordered product
of fermion operators in terms of normal-ordered products and contractions. It may be stated as follows:

A $\hat{\mathcal{T}}$ product of m fermion operators can be transformed into a sum of $\hat{\mathcal{T}}$ products with all possible contractions of $k = 0, 1, \dots, [m/2]$ operator pairs:
\begin{equation}
	\begin{aligned}
		\hat{\boldsymbol{T}}\left[O_{i} O_{j} O_{k} O_{l} \ldots O_{r} O_{s} O_{t}\right]&=
		\hat{\mathcal{N}}\left[O_{i} O_{j} O_{k} O_{l} \ldots O_{r} O_{s} O_{t}\right]
		\\
		&+\hat{\mathcal{N}}\left[\wick{\c1 O_{i} \c1 O_{j} O_{k} \ldots}\right] +\hat{\mathcal{N}}\left[\wick{\c1 O_{i} O_{j} \c1 O_{k} \ldots}\right]+\ldots
		\\
		&+\hat{\mathcal{N}}\left[\wick{\c1 O_{i} \c1 O_{j} \c1 O_{k} \c1 O_{l} \ldots}\right]+\ldots
		\\
		&+\hat{\mathcal{N}}\left[\wick{\c3 O_{i} \c2 O_{j} \c1 O_{k} \ldots \c1 O_{r} \c2 O_{s} \c3 O_{t}}\right] \ldots
	\end{aligned}
\end{equation}

According to the property of the $\hat{\mathcal{N}}$ products, only the fully contracted terms contribute to the expectation value:
\begin{equation}
	\left\langle\Phi_{0}\left|
	\hat{\mathcal{T}}\left[O_{i} O_{j} O_{k} O_{l} \ldots O_{r} O_{s} O_{t}\right]
	\right| \Phi_{0}\right\rangle
	=\hat{\mathcal{N}}\left[\wick{\c3 O_{i} \c2 O_{j} \c1 O_{k} \ldots \c1  O_{r} \c2 O_{s} \c3 O_{t}}\right]+\ldots
\end{equation}

With Wick's theorem, we can finally calculate Green functions by series.
The perturbed Hamiltonian is:
\begin{equation}
	\hat{H}_{I}(t)=\sum w_{r s} c_{r}^{\dagger}(t) c_{s}(t)+\frac{1}{2} \sum V_{u v r s} c_{u}^{\dagger}(t) c_{v}^{\dagger}(t) c_{s}(t) c_{r}(t)
\end{equation}

First Order:
\begin{equation}
	\begin{aligned}
		&i \tilde{G}_{p q}^{(1)}\left(t, t^{\prime}\right)=
		(-i) \frac{1}{2} \sum_{r, s} w_{r s} 
		\int_{-\infty}^{\infty} \mathrm{d} t_{1} e^{-\epsilon\left|t_{1}\right|}
		\left\langle\Phi_{0}\left|
		\hat{\mathcal{T}}\left[
			c_{r}^{\dagger}\left(t_{1}\right) 
			c_{s}\left(t_{1}\right) 
			c_{p}(t)
			c_{q}^{\dagger}\left(t^{\prime}\right)
			\right]
		\right| \Phi_{0}\right\rangle
	\\
		&+(-i) \frac{1}{2} \sum_{u, v, r, s} V_{u v r s} 
		\int_{-\infty}^{\infty} \mathrm{d} t_{1} e^{-\epsilon\left|t_{1}\right|}
		\left\langle\Phi_{0}\left|
		\hat{\mathcal{T}}\left[
			c_{u}^{\dagger}\left(t_{1}\right) 
			c_{v}^{\dagger}\left(t_{1}\right)
			c_{s}\left(t_{1}\right) 
			c_{r}\left(t_{1}\right)
			c_{p}(t)
			c_{q}^{\dagger}\left(t^{\prime}\right)
		\right]
		\right| \Phi_{0}\right\rangle
	\end{aligned}
\end{equation}

Then we calculate the time-ordered products:
\begin{equation}
	\begin{aligned}
		&w_{rs}
		\left\langle\Phi_{0}\left|
		\hat{\mathcal{T}}\left[
			c_{r}^{\dagger}\left(t_{1}\right) 
			c_{s}\left(t_{1}\right) 
			c_{p}(t)
			c_{q}^{\dagger}\left(t^{\prime}\right)
			\right]
		\right| \Phi_{0}\right\rangle
		\\
		=&w_{rs} (G^0_{pr}(t,t_1) G^0_{sq}(t_1,t^{\prime}) - G^0_{rs}(t_1,t_1) G^0_{pq}(t,t^{\prime}))
		\\
		=&w_{pq} G^0_p(t,t_1) G^0_q(t_1,t^{\prime}) - w_{rr} G^0_r(t_1,t_1) \delta_{pq} G^0_{p}(t,t^{\prime})
		\\
		=&A+B
	\end{aligned}
\end{equation}
where $G^0_{pq}(t,t^{\prime}) = \delta_{pq} G^0_{p}(t,t^{\prime})$ is used in the last step

\begin{equation}
	\begin{aligned}
		&V_{uvrs}
		\left\langle\Phi_{0}\left|
		\hat{\mathcal{T}}\left[
			c_{u}^{\dagger}\left(t_{1}\right) 
			c_{v}^{\dagger}\left(t_{1}\right)
			c_{s}\left(t_{1}\right) 
			c_{r}\left(t_{1}\right)
			c_{p}(t)
			c_{q}^{\dagger}\left(t^{\prime}\right)
		\right]
		\right| \Phi_{0}\right\rangle
		\\
		=& -V_{pr[qr]} G^0_{p}(t,t_1) G^0_{r}(t_1,t_1) G^0_{q}(t_1,t^{\prime})
		- V_{rs[rs]} G^0_r(t_1,t_1) G^0_s(t_1,t_1) \delta_{pq} G^0_p(t,t^{\prime})
		\\
		=&C+D
	\end{aligned}
\end{equation}

The idea of feynman diagram is to assign each expression (A, B, C, D) here to a diagram.
The rule is to assign each time (including $t$, $t^{\prime}$ and time appearing as intergration variable) a vertex, to assign each free Green function a line or curve which connects the two vertexs correponding to the time variables of the Green function itself.
The starting time of the Green function in interest is always on the bottom while in end time is always on the top.

In expression A, free Green function first propagate from time $t$ to $t_1$, and then from $t_1$ to $t^{\prime}$, which corresponds to a connected line as is shown in Diagram A.
In expression B, one free Green function propagates from $t$ to $t^{\prime}$, which corresponds to a line, while the other propagates from $t_1$ to itself, which corresponds to a loop (a curve ends at the starting vertex).
The case C and D are also similar, except that $t_1$ has four indexes, which means it should be connected to four lines or curves.

\begin{figure}[ht]
	\centering
	\begin{subfigure}{0.2\textwidth}
		\centering
		\includegraphics[height=3cm]{figures/diagramA.png}
		\caption{Diagram A}
	\end{subfigure}
	\begin{subfigure}{0.2\textwidth}
		\centering
		\includegraphics[height=3cm]{figures/diagramB.png}
		\caption{Diagram B}
	\end{subfigure}
	\begin{subfigure}{0.2\textwidth}
		\centering
		\includegraphics[height=3cm]{figures/diagramC.png}
		\caption{Diagram C}
	\end{subfigure}
	\begin{subfigure}{0.2\textwidth}
		\centering
		\includegraphics[height=3cm]{figures/diagramD.png}
		\caption{Diagram D}
	\end{subfigure}
\end{figure}

Additionally, we find that in the $w_{rs}$ case (A and B), there is always one incoming line and one outcoming line for each intermediate vertex, which is because of one creation operator and one annihilation operator in the expression $W=w_{rs} c_r^{\dagger} c_s$.
In the $V_{uvrs}$ case (C and D), there are always two incoming lines and two outcoming lines for rach intermediate vertex, which is because of two creation operators and two annihilation operators in the expression $V=\frac{1}{2}V_{uvrs} c_u^{\dagger} c_v^{\dagger} c_s c_r$.

Compared with the expressions, the diagrams are very simple, which is one of its advantages.
In fact, Feynman diagram can not only show the structure of expression, but can also restore all the algebraic details.
This means that, instead of doing any algebraic calculation, we can just draw all the possible diagrams and then use some rules to translate to algebraic results.
Then we will derive the rules:

For each vertex, we assign $w_{rs}$ or $V_{uv[rs]}$ depending on whether it's a $w$ vertex or a $V$ vertex:

\hspace{0.2\textwidth}
\begin{minipage}{0.08\textwidth}
	\includegraphics[height=3cm]{figures/vertexW.png}
\end{minipage}
\begin{minipage}{0.2\textwidth}
	$=w_{rs}$
\end{minipage}
\begin{minipage}{0.18\textwidth}
	\includegraphics[height=3cm]{figures/vertexV.png}
\end{minipage}
\begin{minipage}{0.1\textwidth}
	$=V_{uv[rs]}$
\end{minipage}

It is easy to understand the result of $w$ vertex, since $W=w_{rs} c_r^{\dagger} c_s$.
For the $V$ case, $V=\frac{1}{2}V_{uvrs} c_u^{\dagger} c_v^{\dagger} c_s c_r$.
Wick's theorem states all possible of contraction should be included, thus any algebraic expression will have "partners" that the only difference is the exchange of index $u$ and $v$, or $s$ and $r$, or both.
Thus, instead of $\frac{1}{2}V_{uvrs}$, the proper magnitude should be $\frac{1}{2}V_{[uv][rs]}=V_{uv[rs]}$.

However, it is not always the case, since sometimes we will meet double counting.
For example in the double loop in Diagram D, which corresponds to algebraic expression
\begin{equation}
	\left\langle\Phi_{0}\left|
	\hat{\mathcal{T}}\left[
		c_{u}^{\dagger}\left(t_{1}\right) 
		c_{v}^{\dagger}\left(t_{1}\right)
		c_{s}\left(t_{1}\right) 
		c_{r}\left(t_{1}\right)
	\right]
	\right| \Phi_{0}\right\rangle
\end{equation}

Obviously, there is only two contraction schemes, i.e. $u$ with $s$, $v$ with $r$, or $u$ with $r$, $v$ with $s$. 
Thus, we meet double counting here.
Genrally, when two lines are equal, double counting happens.
Thus, the final result should be divided by two for each pair of equal lines.


Each line or curve is assigned by its corresponding Green function, which is also easy to understand.
In the case of higher orders, the expression is 
\begin{equation} \label{greenexp}
\begin{aligned}
	i \tilde{G}^n_{p q}\left(t, t^{\prime}\right)
	=& \frac{(-i)^{n}}{n !} 
	\int_{-\infty}^{\infty} \mathrm{d} t_{1} e^{-\epsilon\left|t_{1}\right|} \ldots \int_{-\infty}^{\infty} \mathrm{d} t_{n} e^{-\epsilon\left|t_{n}\right|}
	\\
	& \left\langle\Phi_{0}\left|
	\hat{\boldsymbol{T}}\left[\hat{H}_{I}\left(t_{1}\right) \ldots \hat{H}_{I}\left(t_{n}\right) c_{p}(t) c_{q}^{\dagger}\left(t^{\prime}\right)\right]\right|
	\Phi_{0}\right\rangle
\end{aligned}
\end{equation}
which has a factor of $\frac{(-i)^n}{n!}$.

However, $t_1 \dots t_n$ are treated on each footing and can also be exchanged when calculting the time-ordered product, thus it will contribute a factor $n!$.
Thus the overall factor is $(-i)^n$.

Last but not least, we need to determine the overall sign of the expression.
Unfortunately, it is impossible to determine the sign just from diagram, thus one must go back to any one of the algebraic expressions and then count how many times one need to commute creation and annihilation operator in the expression of time-ordered product.

Feynman diagram drawed by above rules is refered as Abrikosov diagram.

Then we go back to the calculation of first order Green function.
Let's first focus on expression B and D.
Both parts contain $G^0_{pq}(t,t^{\prime})$, which corresponds to a separate line in Diagram B and D.
Now we introduce the famous linked-cluster theorem, which states that all diagrams contain a separate $G^0_{pq}(t,t^{\prime}$ line will be canceled by the denominator of Eq \ref{greenexp}.
The proof is given in the reference \cite{main}.
Here we take first order Green function as an example.

Up to the first order, Green function can be written as
\begin{equation}
	\begin{aligned}
		i\tilde{G}_{pq}(t,t^{\prime})=&i \delta_{pq} G^0_{p}(t,t^{\prime})
		\\
		+&w_{pq} G^0_p(t,t_1) G^0_q(t_1,t^{\prime}) - w_{rr} G^0_r(t_1,t_1) \delta_{pq} G^0_{p}(t,t^{\prime})
		\\
		-&V_{pr[qr]} G^0_{p}(t,t_1) G^0_{r}(t_1,t_1) G^0_{q}(t_1,t^{\prime})
		- V_{rs[rs]} G^0_r(t_1,t_1) G^0_s(t_1,t_1) \delta_{pq} G^0_p(t,t^{\prime})
		\\
		+&O(2)
	\end{aligned}
\end{equation}
while the denominator is 
\begin{equation}
	\left\langle\Phi_{0}\left|\hat{U}_{\epsilon}(0,-\infty)\right| \Phi_{0}\right\rangle=1
		- w_{rr} G^0_r(t_1,t_1)
		- V_{rs[rs]} G^0_r(t_1,t_1) G^0_s(t_1,t_1)
		+ O(2)
\end{equation}

Thus up to the first order,  the overall Green function reads
\begin{equation}
	i G_{pq}(t,t^{\prime})=i \delta_{pq} G^0_{p}(t,t^{\prime})
	+w_{pq} G^0_p(t,t_1) G^0_q(t_1,t^{\prime})
	-V_{pr[qr]} G^0_{p}(t,t_1) G^0_{r}(t_1,t_1) G^0_{q}(t_1,t^{\prime}) +O(2)
\end{equation}
where all unlinked diagrams are canceled.

As we mentioned before, both the numerator and denominator contains divergent terms, which will be canceled after division.
In fact, the divergent parts are excatly the unlinked diagrams.
Thus, linked-cluster theorem decreases the number of diagrams we need to calculate, and also gurantees the result is finite and thus physical.

Then we analyze the rest terms of first order Green function:
\begin{equation}
	\begin{aligned}
		i G^1_{pq}(t,t^{\prime}) &= w_{pq} G^0_p(t,t_1) G^0_q(t_1,t^{\prime})
		-V_{pr[qr]} G^0_{p}(t,t_1) G^0_{r}(t_1,t_1) G^0_{q}(t_1,t^{\prime})
		\\
		&= G^0_p(t,t_1) G^0_q(t_1,t^{\prime}) (w_{pq}-V_{pr[qr]}G^0_r(t_1,t_1)
		\\
		&= G^0_p(t,t_1) G^0_q(t_1,t^{\prime}) (w_{pq}+V_{pr[qr]}n_r)
		\\
		&=0
	\end{aligned}
\end{equation}
where 
\begin{equation}
	\begin{aligned}
		G^0_r(t_1,t_1)&=G^0_r(t_1,t_1^{+})
		\\
		&=- \left\langle\Phi_{0}\left|
		c_r^{\dagger} c_r
		\right| \Phi_{0}\right\rangle
		\\
		&=-n_r
	\end{aligned}
\end{equation}
is used.

In fact, the diagrams that contain $w$ vertex will always cancel with the diagrams that cantain free Green function line starting and ending with the same $V$ vertex.
According to Eq \ref{greenexp}, $H_I$, which is $W$ and $V$ always appear together.
Thus, any diagram contain $V_{pr[qr]}n_r$ will automatically contain $r_{pq}$.
This result means that we can skip all diagrams with $W$ vertex and that cantain free Green function line starting and ending with the same $V$ vertex at the same time.
Thus, we will not consider $W$ vertex in the following.

Then we summarize the Feynman rules for Abrikosov diagram:
\begin{itemize}
	\item Draw all topologically distinct connected diagrams with $n$ interaction dots and $2n + 1$ directed (solid) free Green’s function starting at the outer vertex $(p, t)$ and ending at the outer vertex $(q, t^{\prime})$.
		At each interaction dot, two Green function start and two end; assign a time argument to each interaction dot.
	\item Attach one-particle indices and time arguments to the free Green function lines; the arrows define the order of the time arguments. Replace the graphical symbols (free Green function lines and interaction dots) by the respective analytical expressions.
	\item Sum over indices and integrate over time arguments of the inner vertices.
	\item The overall phase of an Abrikosov diagram can only be fixed by inspecting one of the Feynman diagrams comprised in the Abrikosov diagram.
		The phase is to be adapted in such a way that this Feynman diagram is reproduced correctly by the Abrikosov expression.
	\item Apply a factor of $\frac{1}{2}$ for each pair of (topologically) equivalent free Green function lines to compensate for double counting of Feynman diagrams. Double counting may arise for other reasons at fourth and higher order, and this possibility must be checked at the level of Feynman diagrams.
\end{itemize}

We calculate the second order Green function as the end of this section:
According to the Feynman diagram, we first draw all possible topologically inequvalent diagrams.
In the case of second order Green function, there is only one diagram:
\begin{figure}[h]
	\centering
	\includegraphics[height=4cm]{figures/order2.png}
	\caption{Diagram of second order Green function}
\end{figure}

Note that the two curves are identical ( start and end with the same vertexs).
Thus, according the Feynman rules, we can easily determine the second order Green function up to a overall sign:
\begin{equation} \label{order2}
	\begin{aligned} 
	G_{p q}^{(2)}\left(t, t^{\prime}\right)=\pm \frac{1}{2} \sum_{r, u, v_{-\infty}} & \int_{-\infty}^{\infty} \mathrm{d} t_{1} \int_{-\infty}^{\infty} \mathrm{d} t_{2} V_{p r[u v]} V_{u v[q r]} 
	\\ 
	& G_{p}^{0}\left(t, t_{1}\right) G_{u}^{0}\left(t_{1}, t_{2}\right) G_{v}^{0}\left(t_{1}, t_{2}\right) G_{r}^{0}\left(t_{2}, t_{1}\right) G_{q}^{0}\left(t_{2}, t^{\prime}\right) 
	\end{aligned}
\end{equation}

To determine the overall sign, we only need to consider the case
\begin{equation}
	\left\langle\Phi_{0}\left|
	\hat{\mathcal{T}}\left[
		\wick{\c4 c_{u}^{\dagger}\left(t_{1}\right) \c3 c_{v}^{\dagger}\left(t_{1}\right) \c2 c_{s}\left(t_{1}\right) \c1 c_{r}\left(t_{1}\right) \c1 c_{i}^{\dagger}\left(t_{2}\right) \c2 c_{j}^{\dagger}\left(t_{2}\right) \c3 c_{l}\left(t_{2}\right) \c1 c_{k}\left(t_{2}\right) \c4 c_{p}(t) \c1 c_{q}^{\dagger}\left(t^{\prime}\right)}
		\right]
		\right| \Phi_{0}\right\rangle
\end{equation}
which gives an overall sign of $-1$, which is canceled with the factor $(-i)^2$.

Thus the overall sign of expression Eq \ref{order2} should be positive.


\input{chapters/floats}
\input{chapters/math}
\input{chapters/citations}
\bibliography{bib/ustc}
\appendix
\input{chapters/complementary}

\backmatter
% !TeX root = ../main.tex

\begin{acknowledgements}
	First of all, I am very grateful to Heidelberg Univerisity (HU) and Univeristy of Science and Technology of China (USTC) for giving me the chance to write the thesis in Germany, and China Scholarship Council for offering me finicial support in this period of time.

	Then I really thank Professor Andreas Dreuw.
	When I first came to this group, he arranged a research project that was very suitable for me according to my research interests, which is ADC.
	This is very helpful for me to understand many fields related to it.
	I am also very grateful to Sebastian and Dirk.
	Sebastian gave me a lot of help in the implementation of the program, giving me a detailed explanation of how qchem works, and answering many of my specific questions about coding.
	With a lot of knowledge, Dirk also gave me many good advice when neither Sebistian nor I can solve some problems.

	I also need to thank many people at USTC.
	Professor Jun Jiang, who serves as my class monitor, supports me a lot when I was still a freshman and had little idea of what I should do and can do in the future.
	He is also the one that took me to the way of theoretical chemistry, which is proved to be the best decision I made in my undergraduate life.
	By working in his group on an application of embedding theory on graphene, I had chance to learn much in theoretical chemistry, which keeps me excited in science and research.
	Yukai, who is currently a PhD student at Princeton Univeristy and one year senior than me at USTC, gave me a lot of constructive suggestions on study and application for U.S. universities, and is also a person that I respect very much for his taste and understanding in physics.

	I also have to mention my best friends Linhao and Shubo.
	Linhao discussed a lot of academic problems with me very frequently and thus helped me a lot on the way to science.
	Shubo shared a lot of his thoughts on study and life with me and helped me be awared of the meaning of my life.
	Both of them helped me a lot to relieve depression when I had difficulties for several times, especially on TOEFL exam and application for U.S. universities.
	I also thank my friends Zheru, Xuchen for a lot of meaningful chattings and discussions with me, Sen, Zihan, Daozheng, Wanying for having worked with me on scientific projects and making great contributions.

	Finally and most importanly, I need to thank my father and mother, who love and support me without any reservation from my birth.
\end{acknowledgements}

\input{chapters/publications}

\end{document}
